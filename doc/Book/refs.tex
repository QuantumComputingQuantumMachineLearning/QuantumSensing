\begin{thebibliography}{99}

\bibitem{Giovannetti2004} V. Giovannetti, S. Lloyd, and L. Maccone, Quantum-Enhanced Measurements: Beating\
 the Standard Quantum Limit,'' Science \textbf{306}, 1330 (2004). \bibitem{Budker2007} D. Budker and M. Ro\
malis, Optical magnetometry,’’ Rev. Mod. Phys. \textbf{79}, 1267 (2007).

\bibitem{Degen2017} C. L. Degen, F. Reinhard, and P. Cappellaro, Quantum sensing,'' Rev. Mod. Phys. \textb\
f{89}, 035002 (2017). \bibitem{Taylor2008} J. M. Taylor \emph{et al.}, High-sensitivity diamond magnetomet\
er based on the NV center in diamond,’’ Nature Phys. \textbf{4}, 810 (2008).

\bibitem{Maze2008} J. R. Maze \emph{et al.}, Nanoscale magnetic sensing with an individual electronic spin\
 in diamond,'' Nature \textbf{455}, 644 (2008). \bibitem{Aslam2023} N. Aslam \emph{et al.}, Quantum sensor\
s for biomedical applications,’’ Nat. Rev. Phys. \textbf{5}, 157 (2023).

\bibitem{Cronin2009} A. D. Cronin, J. Schmiedmayer, and D. E. Pritchard, Optics and interferometry with at\
oms and molecules,'' Rev. Mod. Phys. \textbf{81}, 1051 (2009). \bibitem{Bongs2019} K. Bongs \emph{et al.},\
 Taking atom interferometric quantum sensors from the laboratory to real-world applications,’’ Nat. Rev. P\
hys. \textbf{1}, 731 (2019).

\bibitem{Stray2022} B. Stray \emph{et al.}, Quantum sensing for gravity cartography,'' Nature \textbf{602}\
, 590 (2022). \bibitem{Aspelmeyer2014} M. Aspelmeyer, T. J. Kippenberg, and F. Marquardt, Cavity optomecha\
nics,’’ Rev. Mod. Phys. \textbf{86}, 1391 (2014).



\bibitem{Kay1993} Kay, S. (1993). \textit{Fundamentals of Statistical Signal Processing, Volume I: Estimation Theory}. Prentice Hall PTR. (Classical estimation theory reference.)



\bibitem{Helstrom1976} Helstrom, C. W. (1976). \textit{Quantum Detection and Estimation Theory}. Academic Press, New York. (Foundational text introducing quantum estimation and the QCRB.)



\bibitem{Holevo2011} Holevo, A. S. (2011). \textit{Probabilistic and Statistical Aspects of Quantum Theory}. Edizioni della Normale, Pisa. (Originally published in Russian, 1982. Provides a comprehensive mathematical treatment of quantum statistics and estimation.)



\bibitem{BraunsteinCaves1994} Braunstein, S. L., & Caves, C. M. (1994). Statistical distance and the geometry of quantum states. \textit{Physical Review Letters, 72}(22), 3439–3443. (Introduced the concept of statistical distance and derived the quantum Fisher information in a geometric framework.)



\bibitem{Paris2009} Paris, M. G. A. (2009). Quantum estimation for quantum technology. \textit{International Journal of Quantum Information, 7}(Supp. 1), 125–137. (A review of local quantum estimation theory with formulas for SLD and QFI for various states.)



\bibitem{Giovannetti2011} Giovannetti, V., Lloyd, S., & Maccone, L. (2011). Advances in quantum metrology. \textit{Nature Photonics, 5}(4), 222–229. (Review article on quantum metrology, discussing strategies to beat standard quantum limits.)



\bibitem{Bollinger1996} Bollinger, J. J., Itano, W. M., Wineland, D. J., & Heinzen, D. J. (1996). Optimal frequency measurements with maximally correlated states. \textit{Physical Review A, 54}(6), R4649–R4652. (Proposal of using maximally entangled states (GHZ) for improved spectroscopy.)



\bibitem{Huelga1997} Huelga, S. F., Macchiavello, C., Pellizzari, T., Ekert, A. K., Plenio, M. B., & Cirac, J. I. (1997). Improvement of frequency standards with quantum entanglement. \textit{Physical Review Letters, 79}(20), 3865–3868. (Seminal paper showing that entanglement (GHZ) does not improve clock precision under realistic decoherence beyond a constant factor.)



\bibitem{Ragy2016} Ragy, S., Jarzyna, M., & Demkowicz-Dobrzański, R. (2016). Compatibility in multiparameter quantum metrology. \textit{Physical Review A, 94}(5), 052108. (Analysis of conditions for saturating multi-parameter QCRB and incompatibility issues.)



\bibitem{Demkowicz2012} Demkowicz-Dobrzański, R., Kołodyński, J., & Guţă, M. (2012). The elusive Heisenberg limit in quantum-enhanced metrology. \textit{Nature Communications, 3}, 1063. (Shows that with decoherence or loss, the $1/N$ Heisenberg scaling becomes unattainable asymptotically, often reverting to $1/\sqrt{N}$ scaling.)



\bibitem{Dur2014} D"ur, W., Skotiniotis, M., Fr"owis, F., & Kraus, B. (2014). Improved quantum metrology using quantum error correction. \textit{Physical Review Letters, 112}(8), 080801. (Example of using error-correcting codes to combat decoherence during sensing, preserving Heisenberg scaling.)



\end{thebibliography}


