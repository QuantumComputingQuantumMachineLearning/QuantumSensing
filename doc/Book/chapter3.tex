\chapter{Quantum Mechanics}
\section{Introduction}
%\section{Qubits and Hilbert Space}
\section{Density Operators and Mixed States}


\paragraph{Density matrix}

Using the above linear expansion we can now define the density matrix of the state $\vert \psi\rangle$ as the outer product
\[
\bm{\rho}=\vert \psi \rangle\langle \psi \vert = \alpha\alpha^* \vert 0 \rangle\langle 0\vert+\alpha\beta^* \vert 0 \rangle\langle 1\vert+\beta\alpha^* \vert 1 \rangle\langle 0\vert+\beta\beta^* \vert 1 \rangle\langle 1\vert,
\]
which leads to
\[
\bm{\rho}=\begin{bmatrix} \alpha\alpha^* & \alpha\beta^*\\ \beta\alpha^* & \beta\beta^*\end{bmatrix}.
\]

Finally, we note that the trace of the density matrix is simply given by unity
\[
\mathrm{tr}\bm{\rho}=\alpha\alpha^* +\beta\beta^*=1.
\]



\paragraph{Measurements}

The probability of a measurement on a quantum system giving a certain
result is determined by the weight of the relevant basis state in the
state vector. After the measurement, the system is in a state that
corresponds to the result of the measurement. The operators and
gates discussed below are examples of operations we can perform on
specific states.

We  consider the state
\[
\vert \psi\rangle = \alpha \vert 0 \rangle +\beta \vert 1 \rangle
\]


\paragraph{Definitions of measurements}

\begin{enumerate}
\item A measurement can yield only one of the above states, either $\vert 0\rangle$ or $\vert 1\rangle$.

\item The probability of a measurement resulting in $\vert 0\rangle$ is $\alpha^*\alpha = \vert \alpha \vert^2$.

\item The probability of a measurement resulting in $\vert 1\rangle$ is $\beta^*\beta = \vert \beta \vert^2$.

\item And we note that the sum of the outcomes gives $\alpha^*\alpha+\beta^*\beta=1$ since the two states are normalized.
\end{enumerate}


After the measurement, the state of the system is the state associated with the result of the measurement.

We have already encountered the projection operators $P$ and $Q$. Let
us now look at other types of operations we can make on qubit states.


\paragraph{Different operators and gates}

In quantum computing, the so-called Pauli matrices, and other simple
$2\times 2$ matrices, play an important role, ranging from the setup
of quantum gates to a rewrite of creation and annihilation operators
and other quantum mechanical operators. Let us start with the familiar
Pauli matrices and remind ourselves of some of their basic properties.

Assume we operate with $\sigma_x$ on our basis state $\vert 0 \rangle$. This gives
\[
\begin{bmatrix} 0 & 1 \\ 1 & 0 \end{bmatrix}\begin{bmatrix} 1 \\ 0 \end{bmatrix}=\begin{bmatrix} 0  \\ 1  \end{bmatrix},
\]
that is we switch from $\vert 0\rangle$ to $\vert 1\rangle$ (often called a spin flip operation) and similary we have
\[
\begin{bmatrix} 0 & 1 \\ 1 & 0 \end{bmatrix}\begin{bmatrix} 0 \\ 1 \end{bmatrix}=\begin{bmatrix} 1  \\ 0  \end{bmatrix}.
\]


\paragraph{More on Pauli matrices}

This matrix plays an important role in quantum computing since we can
link this with the classical \textbf{NOT} operation.  If we send in bit 0,
the \textbf{NOT} gate outputs bit 1 and vice versa. We can use the $\sigma_x$
matrix to implement the quantum mechanical equivalent of a classical
\textbf{NOT} gate. If we input what we could represent as bit 0 in terms of
the basis state $\vert 0\rangle$, operating on this state results in
the state $\vert 1\rangle$, which we in turn can interpret as the
classical bit 1.


\paragraph{Linear superposition}
If we have a linear superposition of these states we obtain
\[
\begin{bmatrix}0 & 1 \\ 1 & 0 \end{bmatrix}\begin{bmatrix}\alpha \\ \beta \end{bmatrix}=\begin{bmatrix}\beta \\ \alpha \end{bmatrix}.
\]

The $\sigma_y$ matrix introduces an imaginary sign, which we will later encounter in terms of so-called phase-shift operations.


\paragraph{The $\sigma_z$ matrix}
The $\sigma_z$ matrix has the following effect
\[
\begin{bmatrix} 1 & 0 \\ 0 & -1 \end{bmatrix}\begin{bmatrix} 1 \\ 0 \end{bmatrix}=\begin{bmatrix} 1  \\ 0  \end{bmatrix},
\]
and 
\[
\begin{bmatrix} 1 & 0 \\ 0 & -1 \end{bmatrix}\begin{bmatrix} 0 \\ 1 \end{bmatrix}=\begin{bmatrix} 0  \\ -1  \end{bmatrix},
\]
which we will also link with a specific phase-shift.



\paragraph{Measurements}

The probability of a measurement on a quantum system giving a certain
result is determined by the weight of the relevant basis state in the
state vector. After the measurement, the system is in a state that
corresponds to the result of the measurement. The operators and
gates discussed below are examples of operations we can perform on
specific states.

We  consider the state
\[
\vert \psi\rangle = \alpha \vert 0 \rangle +\beta \vert 1 \rangle
\]


\paragraph{Properties of a measurement}

\begin{enumerate}
\item A measurement can yield only one of the above states, either $\vert 0\rangle$ or $\vert 1\rangle$.

\item The probability of a measurement resulting in $\vert 0\rangle$ is $\alpha^*\alpha = \vert \alpha \vert^2$.

\item The probability of a measurement resulting in $\vert 1\rangle$ is $\beta^*\beta = \vert \beta \vert^2$.

\item And we note that the sum of the outcomes gives $\alpha^*\alpha+\beta^*\beta=1$ since the two states are normalized.
\end{enumerate}


After the measurement, the state of the system is the state associated with the result of the measurement.

We have already encountered the projection operators $P$ and $Q$. Let
us now look at other types of operations we can make on qubit states.


\paragraph{Basic properties of hermitian operators}

The operators we typically encounter in quantum mechanical studies are
\begin{enumerate}
\item Hermitian (self-adjoint) meaning that for example the elements of a Hermitian matrix $\bm{U}$ obey $u_{ij}=u_{ji}^*$.

\item Unitary $\bm{U}\bm{U}^{\dagger}=\bm{U}^{\dagger}\bm{U}=\bm{I}$, where $\bm{I}$ is the unit matrix

\item The operator $\bm{U}$ and its self-adjoint commute (often labeled as normal operators), that is  $[\bm{U},\bm{U}^{\dagger}]=0$. An operator is \textbf{normal} if and only if it is diagonalizable. A Hermitian operator is normal.
\end{enumerate}


Unitary operators in a Hilbert space preserve the norm and orthogonality. If $\bm{U}$ is a unitary operator acting on a state $\vert \psi_j\rangle$, the action of

\[
\vert \phi_i\rangle=\bm{U}\vert \psi_j\rangle,
\]
preserves both the norm and orthogonality, that is $\langle \phi_i \vert \phi_j\rangle=\langle \psi_i \vert \psi_j\rangle=\delta_{ij}$, as discussed earlier.


\paragraph{The Pauli matrices again}

As example, consider the Pauli matrix $\sigma_x$. We have already seen that this matrix is a unitary matrix. Consider then an orthogonal and normalized basis $\vert 0\rangle^{\dagger} =\begin{bmatrix} 1 {\&} 0\end{bmatrix}$ and $\vert 1\rangle^{\dagger} =\begin{bmatrix} 0 {\&} 1\end{bmatrix}$ and a state which is a linear superposition of these two basis states

\[
\vert \psi_a\rangle=\alpha_0\vert 0\rangle +\alpha_1\vert 1\rangle.
\]

A new state $\vert \psi_b\rangle$ is given by
\[
\vert \psi_b\rangle=\sigma_x\vert \psi_a\rangle=\alpha_0\vert 1\rangle +\alpha_1\vert 0\rangle.
\]


\paragraph{Spectral Decomposition}

An important technicality which we will use in the discussion of
density matrices, entanglement, quantum entropies and other properties
is the so-called spectral decomposition of an operator.

Let $\vert \psi\rangle$ be a vector in a Hilbert space of dimension $n$ and a hermitian operator $\bm{A}$ defined in this
space. Assume $\vert \psi\rangle$ is an eigenvector of $\bm{A}$ with eigenvalue $\lambda$, that is

\[
\bm{A}\vert \psi\rangle = \lambda\vert \psi\rangle = \lambda \bm{I}\vert \psi \rangle,
\]
where we used $\bm{I}\vert \psi \rangle = 1 \vert \psi \rangle$.
Subtracting the right hand side from the left hand side gives
\[
\left[\bm{A}-\lambda \bm{I}\right]\vert \psi \rangle=0,
\]

which has a nontrivial solution only if the determinant
$\mathrm{det}(\bm{A}-\lambda\bm{I})=0$.


\paragraph{ONB again and again}

We define now an orthonormal basis $\vert i \rangle =\{\vert 0
\rangle, \vert 1\rangle, \dots, \vert n-1\rangle$ in the same Hilbert
space. We will assume that this basis is an eigenbasis of $\bm{A}$ with eigenvalues $\lambda_i$

We expand a new vector using this eigenbasis of $\bm{A}$
\[
\vert \psi \rangle = \sum_{i=0}^{n-1}\alpha_i\vert i\rangle,
\]
with the normalization condition $\sum_{i=0}^{n-1}\vert \alpha_i\vert^2$.
Acting with $\bm{A}$ on this new state results in

\[
\bm{A}\vert \psi \rangle = \sum_{i=0}^{n-1}\alpha_i\bm{A}\vert i\rangle=\sum_{i=0}^{n-1}\alpha_i\lambda_i\vert i\rangle.
\]


\paragraph{Projection operators}

If we then use that the outer product of any state with itself defines a projection operator we have the projection operators
\[
\bm{P}_{\psi} = \vert \psi\rangle\langle \psi\vert,
\]
and
\[
\bm{P}_{j} = \vert j\rangle\langle j\vert,
\]
we have that 
\[
\bm{P}_{j}\vert \psi\rangle=\vert j\rangle\langle j\vert\sum_{i=0}^{n-1}\alpha_i\vert i\rangle=\sum_{i=0}^{n-1}\alpha_i\vert j\rangle\langle j\vert i\rangle.
\]


\paragraph{Further manipulations}

This results in
\[
\bm{P}_{j}\vert \psi\rangle=\alpha_j\vert j\rangle,
\]
since $\langle j\vert i\rangle$.
With the last equation we can rewrite
\[
\bm{A}\vert \psi \rangle = \sum_{i=0}^{n-1}\alpha_i\lambda_i\vert i\rangle=\sum_{i=0}^{n-1}\lambda_i\bm{P}_i\vert \psi\rangle,
\]
from which we conclude that
\[
\bm{A}=\sum_{i=0}^{n-1}\lambda_i\bm{P}_i.
\]


\paragraph{Spectral decomposition}

This is the spectral decomposition of a hermitian and normal
operator. It is true for any state and it is independent of the
basis. The spectral decomposition can in turn be used to exhaustively
specify a measurement, as we will see in the next section.

As an example, consider two states $\vert \psi_a\rangle$ and $\vert
\psi_b\rangle$ that are eigenstates of $\bm{A}$ with eigenvalues
$\lambda_a$ and $\lambda_b$, respectively. In the diagonalization
process we have obtained the coefficients $\alpha_0$, $\alpha_1$,
$\beta_0$ and $\beta_1$ using an expansion in terms of the orthogonal
basis $\vert 0\rangle$ and $\vert 1\rangle$.


\paragraph{Explicit results}

We have then

\[
\vert \psi_a\rangle = \alpha_0\vert 0\rangle+\alpha_1\vert 1\rangle,
\]
and
\[
\vert \psi_b\rangle = \beta_0\vert 0\rangle+\beta_1\vert 1\rangle,
\]
with corresponding projection operators

\[
\bm{P}_a=\vert \psi_a\rangle \langle \psi_a\vert = \begin{bmatrix} \vert \alpha_0\vert^2 &\alpha_0\alpha_1^* \\
                                                                   \alpha_1\alpha_0^* & \vert \alpha_1\vert^* \end{bmatrix},
\]    
and
\[
\bm{P}_b=\vert \psi_b\rangle \langle \psi_b\vert = \begin{bmatrix} \vert \beta_0\vert^2 &\beta_0\beta_1^* \\
                                                                   \beta_1\beta_0^* & \vert \beta_1\vert^* \end{bmatrix}.
\]


\paragraph{The spectral decomposition}

The results from the previous slide gives us
the following spectral decomposition of $\bm{A}$
\[
\bm{A}=\lambda_a \vert \psi_a\rangle \langle \psi_a\vert+\lambda_b \vert \psi_b\rangle \langle \psi_b\vert,
\]
which written out in all its details reads
\[
\bm{A}=\lambda_a\begin{bmatrix} \vert \alpha_0\vert^2 &\alpha_0\alpha_1^* \\
                                                                   \alpha_1\alpha_0^* & \vert \alpha_1\vert^* \end{bmatrix} +\lambda_b\begin{bmatrix} \vert \beta_0\vert^2 &\beta_0\beta_1^* \\
                                                                   \beta_1\beta_0^* & \vert \beta_1\vert^* \end{bmatrix}.
\]


\paragraph{Bloch sphere}

Classically, in a binary system, the bits take only two distinct
values, either $0$ or $1$.  The quantum mechanical counterpart is
given by two state vectors (our simple computational basis)
$\vert 0 \rangle$ and $\vert 1\rangle $ which can be used to realize the
superposition
\[
\vert \psi \rangle = \alpha \vert 0 \rangle +\beta\vert 1\rangle, 
\]
which can be represented using the so-called Bloch sphere, depicted on the next slide (best seen using the jupyter-notebook).


\paragraph{Meet the Bloch sphere}
The Bloch shere gives a vialable way to visualize a qubit and itsv possible realizations in terms of the angles $0\le \theta \le \pi$ and
$0\le \phi \le 2\pi$. 




\begin{Verbatim}[numbers=none,fontsize=\fontsize{9pt}{9pt},baselinestretch=0.95]
import numpy as np
from qiskit.visualization import plot_bloch_vector
plot_bloch_vector([0,1,0], title="New Bloch Sphere")

\end{Verbatim}

You can use spherical coordinates instead of cartesian ones.


\begin{Verbatim}[numbers=none,fontsize=\fontsize{9pt}{9pt},baselinestretch=0.95]
plot_bloch_vector([1, np.pi/2, np.pi/3], coord_type='spherical')

\end{Verbatim}

Using the Bloch sphere representation of the qubit $\vert \psi \rangle = \alpha \vert 0 \rangle +\beta\vert 1\rangle$, we can rewrite it as
\[
\vert \psi \rangle = \cos{(\frac{\theta}{2})} \vert 0 \rangle +\sin{(\frac{\theta}{2})}\exp{(\imath\phi)}\vert 1\rangle, 
\]


\paragraph{Bloch sphere exercise}

Determine the Bloch sphere angles $\theta$ and $\phi$ for the eigenstates of each Pauli matrix (see lectures from last week for the definition of Pauli matrices).


\paragraph{Measurements}

Armed with the spectral decomposition, we are now ready to discuss how
to compute measurements of observables.  When we make a measurement,
quantum mechanics postulates that mutually exclusive measurement
outcomes correspond to orthogonal projection operators.

We assume now we can contruct a series of such orthogonal operators based on $\vert i \rangle \in \{\vert 0\rangle, \vert 1\rangle,\dots, \vert n-1\rangle$ computational basis states. These projection operators $\bm{P}_0,\bm{P}_1,\dots,\bm{P}_{n-1}$ are all idempotent and sum to one
\[
\sum_{i=0}^{n-1}\bm{P}_i=\bm{I}.
\]


\paragraph{Qubit example}

As an example, consider the basis of two qubits $\vert 0\rangle$ and $\vert 1\rangle$ with the correspong sum
\[
\sum_{i=0}^{1}\bm{P}_i=\begin{bmatrix} 1 & 0 \\ 0 & 1\end{bmatrix}.
\]
Based on the spectral decomposition discussed above, we can define the probability of eigenvalue $\lambda_i$ as
\[
\mathrm{Prob}(\lambda_i) = \vert \bm{P}_i\vert \psi\rangle\vert^2,
\]
where $\vert \psi_a\rangle$ is a quantum state representing the system prior to a specific measurement.


\paragraph{Total probability}

We can rewrite this as 
\[
\mathrm{Prob}(\lambda_i) = \langle \psi\vert \bm{P}_i^{\dagger}\bm{P}_i\vert \psi\rangle =\langle \psi\vert \bm{P}_i\vert \psi\rangle.
\]
The total probability for all measurements is the sum overt all probabilities
\[
\sum_{i=0}^{n-1}\mathrm{Prob}(\lambda_i)=1.
\]
We can in turn define the post-measurement normalized pure quantum state as, for the specific outcome $\lambda_i$, as
\[
\vert \psi'\rangle = \frac{\bm{P}_i\vert \psi\rangle}{\sqrt{\langle \psi \vert \bm{P}_i\vert \psi\rangle}}. 
\]


\paragraph{Binary example system}

As an example, consider the binary system states $\vert 0\rangle$ and $\vert 1\rangle$ with corresponding projection operators
\[
\bm{P}_0 =\vert 0 \rangle \langle 0\vert,
\]
and 
\[
\bm{P}_1 =\vert 1 \rangle \langle 1\vert,
\]
with the properties

\[
\sum_{i=0}^1\bm{P}_i^{\dagger}\bm{P}_1=\bm{I},
\]

\[
\bm{P}_0^{\dagger}\bm{P}_0=\bm{P}_0^2=\bm{P}_0,
\]
and
\[
\bm{P}_1^{\dagger}\bm{P}_1=\bm{P}_1^2=\bm{P}_1.
\]


\paragraph{Superposition of states}

Assume thereafter that we have a state $\vert \psi\rangle$ which is a superposition of the above two qubit states
\[
\vert \psi \rangle = \alpha\vert 0 \rangle + \beta \vert 1\rangle.
\]
The probability of finding either $\vert 0\rangle$ or $\vert 1\rangle$ is then
\[
\bm{P}_{\psi(0)}=\langle \psi\vert \bm{P}_0^{\dagger}\bm{P}_0\vert \psi\rangle=\vert \alpha\vert^2,
\]
and similarly we have 
\[
\bm{P}_{\psi(1)}=\langle \psi\vert \bm{P}_1^{\dagger}\bm{P}_1\vert \psi\rangle=\vert \beta\vert^2.
\]

\paragraph{More derivations}

If we set 
\[
\vert \psi \rangle = \frac{1}{\sqrt{2}}\left(\vert 0 \rangle + \vert 1\rangle\right),
\]
we have $\vert \alpha\vert^2=\vert \beta\vert^2=1/2$. In general for this system we have
\[
\vert \psi'_0\rangle = \frac{\bm{P}_0\vert \psi\rangle}{\sqrt{\langle \psi \vert \bm{P}_0\vert \psi\rangle}}=\frac{\alpha}{\vert \alpha\vert}\vert 0 \rangle,
\]
and
\[
\vert \psi'_1\rangle = \frac{\bm{P}_1\vert \psi\rangle}{\sqrt{\langle \psi \vert \bm{P}_1\vert \psi\rangle}}=\frac{\beta}{\vert \beta\vert}\vert 1 \rangle. 
\]


\paragraph{Final result}
In general we have that 
\[
\bm{P}_{\psi(x)}=\langle \psi\vert \bm{P}_x^{\dagger}\bm{P}_x\vert \psi\rangle,,
\]
which we can rewrite as
\[
\mathrm{Prob}(\psi(x))=\mathrm{Tr}\left[\bm{P}_x^{\dagger}\bm{P}_x\vert \psi\rangle\langle \psi\vert\right].
\]


\paragraph{Example}

The last equation can be understood better through the following example with a state $\vert \psi\rangle$

\[
\vert \psi \rangle = \alpha \vert 0\rangle+\beta \vert 1\rangle,
\]
which results in a projection operator
\[
\vert \psi \rangle\langle \psi\vert = \begin{bmatrix} \vert \alpha \vert^2 & \alpha\beta^* \\ \alpha^*\beta & \vert\beta\vert^2\end{bmatrix}.
\]


\paragraph{Computing matrix products}
We have that
\[
\bm{P}_0^{\dagger}\bm{P}_0=\bm{P}_0=\begin{bmatrix} 1 & 0 \\ 0 & 0\end{bmatrix},
\]
and computing the matrix product $\bm{P}_0\vert\psi\rangle\langle \psi\vert$ gives
\[
\bm{P}_0\vert\psi\rangle\langle \psi\vert=\begin{bmatrix} 1 & 0 \\ 0 & 0\end{bmatrix}\begin{bmatrix} \vert \alpha \vert^2 & \alpha\beta^* \\ \alpha^*\beta & \vert\beta\vert^2\end{bmatrix}=\begin{bmatrix} \vert \alpha \vert^2 & \alpha\beta^* \\ 0 & 0\end{bmatrix}.
\]


\paragraph{Taking the trace}

Taking the trace of the above matrix, that is computing
\[
\mathrm{Prob}(\psi(0))=\mathrm{Tr}\left[\bm{P}_0^{\dagger}\bm{P}_0\vert \psi\rangle\langle \psi\vert\right]=\vert \alpha\vert^2,
\]
we obtain the same results as the one we had earlier by computing
the probabliblity for $0$ given by the expression
\[
\bm{P}_{\psi(0)}=\langle \psi\vert \bm{P}_0^{\dagger}\bm{P}_0\vert \psi\rangle=\vert \alpha\vert^2.
\]


\paragraph{Outcome probability}

It is straight forward to show that
\[
\mathrm{Prob}(\psi(1))=\mathrm{Tr}\left[\bm{P}_1^{\dagger}\bm{P}_1\vert \psi\rangle\langle \psi\vert\right]=\vert \beta\vert^2,
\]
which we also could have obtained by computing
\[
\bm{P}_{\psi(1)}=\langle \psi\vert \bm{P}_1^{\dagger}\bm{P}_1\vert \psi\rangle=\vert \beta\vert^2.
\]


\paragraph{Extending the expressions}

We can now extend these expressions to the complete ensemble of measurements. Using the spectral decomposition we have that the probability of an outcome $p(x)$ is
\[
p(x)=\sum_{i=0}^{n-1}p_i\bm{P}_{\psi_i(x)},
\]
where $p_i$ are the probabilities of a specific outcome. 

With these prerequisites we are now ready to introduce the density  matrices, or density operators.


\paragraph{Density matrices/operators}

The last equation can be rewritten as 

\[
p(x)=\sum_{i=0}^{n-1}p_i\bm{P}_{\psi_i(x)}=\sum_{i=0}^{n-1}p_i\mathrm{Tr}\left[\bm{P}_x^{\dagger}\bm{P}_x\vert \psi_i\rangle\langle \psi_i\vert\right],
\]
and we define the \textbf{density matrix/operator} as
\[
\rho=\sum_{i=0}^{n-1}p_i\vert \psi_i\rangle\langle \psi_i\vert,
\]
we can rewrite the first equation above as 
\[
p(x)=\mathrm{Tr}\left[\bm{P}_x^{\dagger}\bm{P}_x\rho\right].
\]
If we can define the state of a system in terms of the density matrix, the probability of a specific outcome is then given by
\[
p(x)_{\rho}=\mathrm{Tr}\left[\bm{P}_x^{\dagger}\bm{P}_x\rho\right].
\]


\paragraph{Properties of density matrices}

A density matrix in a Hilbert space with $n$ states has the following properties (which we state without proof)
\begin{enumerate}
\item There exists a probability $p_i\geq 0$ with $\sum_ip_i=1$,

\item There exists an orthonormal basis $\psi_i$ such that we can define $\rho=\sum_ip_i\vert\psi_i\rangle\langle \psi_i\vert$,

\item We have $0 \leq \rho^2\leq 1$ and

\item The norm $\vert\vert \rho \vert\vert_2\leq 1$.
\end{enumerate}


With the density matrix we can also define the state the system collapses to after a measurement, namely

\[
\rho'_x=\frac{\bm{P}_x\rho\bm{P}_x^{\dagger}}{\mathrm{Tr}[\bm{P}_x^{\dagger}\bm{P}_x\rho]}.
\]


\paragraph{First entanglement encounter, two qubit system}

We define a system that can be thought of as composed of two subsystems
$A$ and $B$. Each subsystem has computational basis states

\[
\vert 0\rangle_{\mathrm{A,B}}=\begin{bmatrix} 1 & 0\end{bmatrix}^T \hspace{1cm} \vert 1\rangle_{\mathrm{A,B}}=\begin{bmatrix} 0 & 1\end{bmatrix}^T.
\]
The subsystems could represent single particles or composite many-particle systems of a given symmetry.


\paragraph{Computational basis}

This leads to the many-body computational basis states

\[
\vert 00\rangle = \vert 0\rangle_{\mathrm{A}}\otimes \vert 0\rangle_{\mathrm{B}}=\begin{bmatrix} 1 & 0 & 0 &0\end{bmatrix}^T,
\]
and
\[
\vert 01\rangle = \vert 0\rangle_{\mathrm{A}}\otimes \vert 1\rangle_{\mathrm{B}}=\begin{bmatrix} 0 & 1 & 0 &0\end{bmatrix}^T,
\]
and
\[
\vert 10\rangle = \vert 1\rangle_{\mathrm{A}}\otimes \vert 0\rangle_{\mathrm{B}}=\begin{bmatrix} 0 & 0 & 1 &0\end{bmatrix}^T,
\]
and finally
\[
\vert 11\rangle = \vert 1\rangle_{\mathrm{A}}\otimes \vert 1\rangle_{\mathrm{B}}=\begin{bmatrix} 0 & 0 & 0 &1\end{bmatrix}^T.
\]


\paragraph{Bell states}

The above computational basis states, which define an ONB, can in turn
be used to define another ONB. As an example, consider the so-called
Bell states

\[
\vert \Phi^+\rangle = \frac{1}{\sqrt{2}}\left[\vert 00\rangle +\vert 11\rangle\right]=\frac{1}{\sqrt{2}}\begin{bmatrix} 1 \\ 0 \\ 0 \\ 1\end{bmatrix},
\]

\[
\vert \Phi^-\rangle = \frac{1}{\sqrt{2}}\left[\vert 00\rangle -\vert 11\rangle\right]=\frac{1}{\sqrt{2}}\begin{bmatrix} 1 \\ 0 \\ 0 \\ -1\end{bmatrix},
\]


\paragraph{The next two}

\[
\vert \Psi^+\rangle = \frac{1}{\sqrt{2}}\left[\vert 10\rangle +\vert 01\rangle\right]=\frac{1}{\sqrt{2}}\begin{bmatrix} 0 \\ 1 \\ 1 \\ 0\end{bmatrix},
\]

and

\[
\vert \Psi^-\rangle = \frac{1}{\sqrt{2}}\left[\vert 10\rangle -\vert 01\rangle\right]=\frac{1}{\sqrt{2}}\begin{bmatrix} 0 \\ 1 \\ -1 \\ 0\end{bmatrix}.
\]
It is easy to convince oneself that these states also form an orthonormal basis. 


\paragraph{Measurement}

Measuring one of the qubits of one of the above Bell states,
automatically determines, as we will see below, the state of the
second qubit. To convince ourselves about this, let us assume we perform a measurement on the qubit in system $A$ by introducing the projections with outcomes $0$ or $1$ as

\[
\bm{P}_0=\vert 0\rangle\langle 0\vert_A\otimes \bm{I}_B=\begin{bmatrix} 1 & 0\\ 0 & 0\end{bmatrix}\otimes\begin{bmatrix} 1& 0 \\ 0 & 1\end{bmatrix}=\begin{bmatrix} 1 & 0 & 0 & 0 \\ 0 & 1 & 0 & 0 \\ 0 & 0 & 0 & 0 \\ 0 & 0 & 0 & 0\end{bmatrix},
\]
for the projection of the $\vert 0 \rangle$ state in system $A$ and similarly
\[
\bm{P}_1=\vert 1\rangle\langle 1\vert_A\otimes \bm{I}_B=\begin{bmatrix} 0 & 0\\ 0 & 1\end{bmatrix}\otimes\begin{bmatrix} 1& 0 \\ 0 & 1\end{bmatrix}=\begin{bmatrix} 0 & 0 & 0 & 0 \\ 0 & 0 & 0 & 0 \\ 0 & 0 & 1 & 0 \\ 0 & 0 & 0 & 1\end{bmatrix},
\]
for the projection of the $\vert 1 \rangle$ state in system $A$.


\paragraph{Probability of  outcome}

We can then calculate the probability for the various outcomes by
computing for example the probability for measuring qubit $0$ 

\[
\langle \Phi^+\vert \bm{P}_0\vert \Phi^+\rangle = \frac{1}{2} \left[\langle 00\vert +\langle 11\vert\right]\begin{bmatrix} 1 & 0 & 0 & 0 \\ 0 & 1 & 0 & 0 \\ 0 & 0 & 0 & 0 \\ 0 & 0 & 0 & 0\end{bmatrix}\left[\vert 00\rangle +\vert 11\rangle\right]=\frac{1}{2}.
\]
Similarly, we obtain
\[
\langle \Phi^+\vert \bm{P}_1\vert \Phi^+\rangle = \frac{1}{2}\left[\langle 00\vert +\langle 11\vert\right]\begin{bmatrix} 0 & 0 & 0 & 0 \\ 0 & 0 & 0 & 0 \\ 0 & 0 & 1 & 0 \\ 0 & 0 & 0 & 1\end{bmatrix}\left[\vert 00\rangle +\vert 11\rangle\right]=\frac{1}{2}.
\]


\paragraph{States after measurement}
After the above measurements the system is in the states

\[
\vert \Phi'_0 \rangle = \sqrt{2}\left[\vert 0\rangle\langle 0\vert_A\otimes \bm{I}_B\right]\vert\Phi^+\rangle=\vert 00\rangle,
\]
and 
\[
\vert \Phi'_1 \rangle = \sqrt{2}\left[\vert 1\rangle\langle 1\vert_A\otimes \bm{I}_B\right]\vert\Phi^+\rangle=\vert 11\rangle.
\]

We see from the last two equations that the state of the second qubit
is determined even though the measurement has only taken place locally
on system $A$.


\paragraph{Other states}

If we on the other hand consider a state like

\[
\vert 00 \rangle = \vert 0\rangle_A\otimes \vert 0\rangle_B,
\]
this is a pure \textbf{product} state of the single-qubit, or single-particle
states, of two qubits (particles) in system $A$ and system $B$,
respectively. We call such a state for a \textbf{pure product state}.  Quantum states
that cannot be written as a mixture of other states are called pure
quantum states or just product states, while all other states are called mixed quantum states.


\paragraph{More on Bell states}
A state like one of the Bell states (where we introduce the subscript $AB$ to indicate that the state is composed of single states from two subsystem)
\[
\vert \Phi^+\rangle = \frac{1}{\sqrt{2}}\left[\vert 00\rangle_{AB} +\vert 11\rangle_{AB}\right],
\]
is on the other hand a mixed state and we cannot determine whether system $A$ is in a state $0$ or $1$. The above state is a superposition of the states $\vert 00\rangle_{AB}$ and $\vert 11\rangle_{AB}$ and it is not possible to determine individual states of systems $A$ and $B$, respectively.


\paragraph{Entanglement}

We say that the state is entangled. This yields the following
definition of entangled states: a pure bipartite state $\vert
\psi\rangle_{AB}$ is entangled if it cannot be written as a product
state $\vert\psi\rangle_{A}\otimes\vert\phi\rangle_B$ for any choice
of the states $\vert\psi\rangle_{A}$ and $\vert\phi\rangle_B$. Otherwise we say the state is separable.


\paragraph{Examples of entanglement}

As an example, considere an ansatz for the ground state of the helium
atom with two electrons in the lowest $1s$ state (hydrogen-like
orbits) and with spin $s=1/2$ and spin projections $m_s=-1/2$ and
$m_s=1/2$.  The two single-particle states are given by the tensor
products of their spatial $1s$ single-particle states
$\vert\phi_{1s}\rangle$ and and their spin up or spin down spinors
$\vert\xi_{sm_s}\rangle$. The ansatz for the ground state is given by a Slater
determinant with total orbital momentum $L=l_1+l_2=0$ and totalt spin
$S=s_1+s_2=0$, normally labeled as a spin-singlet state.


\paragraph{Ground state of helium}
This ansatz
for the ground state is then written as, using the compact notations
\[
\vert \psi_{i}\rangle = \vert\phi_{1s}\rangle_i\otimes \vert \xi\rangle_{s_im_{s_i}}=\vert 1s,s,m_s\rangle_i,  \]
with $i$ being electron $1$ or $2$, and the tensor product of the two single-electron states as
$\vert 1s,s,m_s\rangle_1\vert 1s,s,m_s\rangle_2=\vert 1s,s,m_s\rangle_1\otimes \vert 1s,s,m_s\rangle_2$, we arrive at
\[
\Psi(\bm{r}_1,\bm{r}_2;s_1,s_2)=\frac{1}{\sqrt{2}}\left[\vert 1s,1/2,1/2\rangle_1\vert 1s,1/2,-1/2\rangle_2-\vert 1s,1/2,-1/2\rangle_1\vert 1s,1/2,1/2\rangle_2\right].
\]
This is also an example of a state which cannot be written out as a pure state. We call this for an entangled state as well.


\paragraph{Maximally entangled}

A so-called maximally entangled state for a bipartite system has equal  probability amplitudes
\[
\vert \Psi \rangle = \frac{1}{\sqrt{d}}\sum_{i=0}^{d-1}\vert ii\rangle.
\]

We call a bipartite state composed of systems $A$ and $B$ (these
systems can be single-particle systems, or single-qubit systems
representing low-lying states of complicated many-body systems) for
separable if its density matrix $\rho_{AB}$ can be written out as the
tensor product of the individual density matrices $\rho_A$ and
$\rho_B$, that is we have for a given probability distribution $p_i$

\[
\rho_{AB}=\sum_ip_i\rho_A(i)\otimes \rho_B(i).
\]



\section{Time Evolution and Measurement}
\section{Conclusion}




The time-dependent Schrödinger equation (or equation of motion) reads

\[
\imath \hbar\frac{\partial }{\partial t}|\Psi_S(t)\rangle = \hat{H}\Psi_S(t)\rangle,
\]

where the subscript $S$ stands for Schrödinger here.
A formal solution is given by 

\[
|\Psi_S(t)\rangle = \exp{(-\imath\hat{H}(t-t_0)/\hbar)}|\Psi_S(t_0)\rangle.
\]


The Hamiltonian $\hat{H}$ is hermitian and the exponent represents a unitary 
operator with an operation carried out on the wave function at a time $t_0$.

The exponential term $\exp{(-\imath\hat{H}(t-t_0)/\hbar)}$ is our
unitary transformation $U(t,t_0)$ and we have

\[
|\Psi_S(t)\rangle = \exp{(-\imath\hat{H}(t-t_0)/\hbar)}|\Psi_S(t_0)\rangle=U(t,t_0)|\Psi_S(t_0)\rangle.
\]


%===== Interaction picture  =====

Our Hamiltonian is normally written out as the sum of an unperturbed part $\hat{H}_0$ and an interaction part $\hat{H}_I$, that is

\[
\hat{H}=\hat{H}_0+\hat{H}_I.
\]

In general we have $[\hat{H}_0,\hat{H}_I]\ne 0$ since $[\hat{T},\hat{V}]\ne 0$.
We wish now to define a unitary transformation in terms of $\hat{H}_0$ by defining

\[
|\Psi_I(t)\rangle = \exp{(\imath\hat{H}_0t/\hbar)}|\Psi_S(t)\rangle,
\]

which is again a unitary transformation carried out now at the time $t$ on the 
wave function in the Schrödinger picture. 



We can easily find the equation of motion by taking the time derivative

\[
\imath \hbar\frac{\partial }{\partial t}|\Psi_I(t)\rangle = -\hat{H}_0\exp{(\imath\hat{H}_0t/\hbar)}\Psi_S(t)\rangle+\exp{(\imath\hat{H}_0t/\hbar)}
\imath \hbar\frac{\partial }{\partial t}\Psi_S(t)\rangle.
\]


Using the definition of the Schrödinger equation, we can rewrite the last equation as 

\[
\imath \hbar\frac{\partial }{\partial t}|\Psi_I(t)\rangle = \exp{(\imath\hat{H}_0t/\hbar)}\left[-\hat{H}_0+\hat{H}_0+\hat{H}_I\right]\exp{(-\imath\hat{H}_0t/\hbar)}\Psi_I(t)\rangle,
\]

which gives us

\[
\imath \hbar\frac{\partial }{\partial t}|\Psi_I(t)\rangle = \hat{H}_I(t)\Psi_I(t)\rangle,
\]

 with 

\[
\hat{H}_I(t)=
\exp{(\imath\hat{H}_0t/\hbar)}\hat{H}_I\exp{(-\imath\hat{H}_0t/\hbar)}.
\]


The order of the operators is important since $\hat{H}_0$ and $\hat{H}_I$ do generally not commute.
The expectation value of
an arbitrary operator in the interaction picture can now be written as

\[
\langle \Psi'_S(t)|\hat{O}_S|\Psi_S(t)\rangle = 
\langle \Psi'_I(t) |\exp{(\imath\hat{H}_0t/\hbar)}\hat{O}_I
\exp{(-\imath\hat{H}_0t/\hbar)}|\Psi_I(t)\rangle,
\]

and using the definition

\[
\hat{O}_I(t)=
\exp{(\imath\hat{H}_0t/\hbar)}\hat{O}_I\exp{(-\imath\hat{H}_0t/\hbar)},
\]

we obtain

\[
\langle \Psi'_S(t)|\hat{O}_S|\Psi_S(t)\rangle = 
\langle \Psi'_I(t) |\hat{O}_I(t)|\Psi_I(t)\rangle,
\]

stating that a unitary transformation does not change expectation values!



If the take the time derivative of the operator in the interaction picture we arrive at the following equation of motion

\[
\imath \hbar\frac{\partial }{\partial t}\hat{O}_I(t) = \exp{(\imath\hat{H}_0t/\hbar)}\left[\hat{O}_S\hat{H}_0-\hat{H}_0\hat{O}_S\right]\exp{(-\imath\hat{H}_0t/\hbar)}=\left[\hat{O}_I(t),\hat{H}_0\right].
\]

Here we have used the time-independence of the Schrödinger equation
together with the observation that any function of an operator commutes with the operator itself. 



In order to solve the equation of motion equation in the interaction picture, we define a unitary
time-development operator $\hat{U}(t,t')$. Later we will derive its
connection with the linked-diagram theorem, which yields a
linked expression for the actual operator. 
The action of the operator on the wave function is

\[
|\Psi_I(t) \rangle = \hat{U}(t,t_0)|\Psi_I(t_0)\rangle,
\]

with the obvious value $\hat{U}(t_0,t_0)=1$.



The time-development operator $U$ has the
properties that

\[
     \hat{U}^{\dagger}(t,t')\hat{U}(t,t')=\hat{U}(t,t')\hat{U}^{\dagger}(t,t')=1,
\]

which implies that $U$ is unitary

\[
     \hat{U}^{\dagger}(t,t')=\hat{U}^{-1}(t,t').
\]

Further,

\[
    \hat{U}(t,t')\hat{U}(t't'')=\hat{U}(t,t'')
\]

and

\[
    \hat{U}(t,t')\hat{U}(t',t)=1,
\]

which leads to

\[
    \hat{U}(t,t')=\hat{U}^{\dagger}(t',t).
\]



Using our definition of Schrödinger's equation in the interaction picture, we can then construct the operator $\hat{U}$. We have defined

\[
|\Psi_I(t)\rangle = \exp{(\imath\hat{H}_0t/\hbar)}|\Psi_S(t)\rangle,
\]

which can be rewritten as 

\[
|\Psi_I(t)\rangle = \exp{(\imath\hat{H}_0t/\hbar)}\exp{(-\imath\hat{H}(t-t_0)/\hbar)}|\Psi_S(t_0)\rangle,
\]

or

\[
|\Psi_I(t)\rangle = \exp{(\imath\hat{H}_0t/\hbar)}\exp{(-\imath\hat{H}(t-t_0)/\hbar)}\exp{(-\imath\hat{H}_0t_0/\hbar)}|\Psi_I(t_0)\rangle.
\]



From the last expression we can define

\[
\hat{U}(t,t_0)=\exp{(\imath\hat{H}_0t/\hbar)}\exp{(-\imath\hat{H}(t-t_0)/\hbar)}\exp{(-\imath\hat{H}_0t_0/\hbar)}.
\]

It is then easy to convince oneself that the properties defined above are satisfied by the definition of $\hat{U}$. 


We derive the equation of motion for $\hat{U}$ using the above definition.
This results in

\[
\imath \hbar\frac{\partial }{\partial t}\hat{U}(t,t_0) = \hat{H}_I(t)\hat{U}(t,t_0),
\]

which we integrate from $t_0$ to a time $t$ resulting in

\[
\hat{U}(t,t_0)-\hat{U}(t_0,t_0)=\hat{U}(t,t_0)-1=-\frac{\imath}{\hbar}\int_{t_0}^t dt' \hat{H}_I(t')\hat{U}(t',t_0),
\]

which can be rewritten as

\[
\hat{U}(t,t_0)=1-\frac{\imath}{\hbar}\int_{t_0}^t dt' \hat{H}_I(t')\hat{U}(t',t_0).
\]


We can solve this equation iteratively keeping in mind the time-ordering of the operators

\[
\hat{U}(t,t_0)=1-\frac{\imath}{\hbar}\int_{t_0}^t dt' \hat{H}_I(t')+\left(\frac{-\imath}{\hbar}\right)^2\int_{t_0}^t dt'\int_{t_0}^{t'} dt'' \hat{H}_I(t')\hat{H}_I(t'')+\dots
\]

The third term can be written as 

\[
\int_{t_0}^t dt'\int_{t_0}^{t'} dt'' \hat{H}_I(t')\hat{H}_I(t'')=
\frac{1}{2}\int_{t_0}^t dt'\int_{t_0}^{t'} dt'' \hat{H}_I(t')\hat{H}_I(t'')
\]


\[
+\frac{1}{2}\int_{t_0}^t dt''\int_{t''}^{t} dt' \hat{H}_I(t')\hat{H}_I(t'').
\]


We obtain this expression by changing the integration order in the second term
via a change of the integration variables $t'$ and $t''$  in 

\[
\frac{1}{2}\int_{t_0}^t dt'\int_{t_0}^{t'} dt'' \hat{H}_I(t')\hat{H}_I(t'').
\]

We can rewrite the terms which contain the double integral as

\[
\int_{t_0}^t dt'\int_{t_0}^{t'} dt'' \hat{H}_I(t')\hat{H}_I(t'')=
\]


\[
\frac{1}{2}\int_{t_0}^t dt'\int_{t_0}^{t'} dt''\left[\hat{H}_I(t')\hat{H}_I(t'')\Theta(t'-t'')
+\hat{H}_I(t')\hat{H}_I(t'')\Theta(t''-t')\right],
\]

with $\Theta(t''-t')$ being the standard Heavyside or step function. The step function allows us to give a specific time-ordering to the above expression.



With the $\Theta$-function we can rewrite the last expression as 

\[
\int_{t_0}^t dt'\int_{t_0}^{t'} dt'' \hat{H}_I(t')\hat{H}_I(t'')=
\frac{1}{2}\int_{t_0}^t dt'\int_{t_0}^{t'} dt''\hat{T}\left[\hat{H}_I(t')\hat{H}_I(t'')\right],
\]

where $\Hat{T}$ is the so-called time-ordering operator. 




With this definition, we can rewrite the expression for $\hat{U}$ as 

\[
\hat{U}(t,t_0)=\sum_{n=0}^{\infty}\left(\frac{-\imath}{\hbar}\right)^n\frac{1}{n!}
\int_{t_0}^t dt_1\dots \int_{t_0}^t dt_N \hat{T}\left[\hat{H}_I(t_1)\dots\hat{H}_I(t_n)\right]=
\]


\[
\hat{T}\exp{\left[\frac{-\imath}{\hbar}
\int_{t_0}^t dt' \hat{H}_I(t')\right]}.
\]




\subsection{Heisenberg  picture as alternative}


We wish now to define a unitary transformation in terms of $\hat{H}$ by defining

\[
|\Psi_H(t)\rangle = \exp{(\imath\hat{H}t/\hbar)}|\Psi_S(t)\rangle,
\]

which is again a unitary transformation carried out now at the time $t$ on the 
wave function in the Schrödinger picture. If we combine this equation with 
Schrödinger's equation we obtain the following equation of motion

\[
\imath \hbar\frac{\partial }{\partial t}|\Psi_H(t)\rangle = 0,
\]

meaning that $|\Psi_H(t)\rangle$ is time independent. An operator in this picture is defined as

\[
\hat{O}_H(t)=
\exp{(\imath\hat{H}t/\hbar)}\hat{O}_S\exp{(-\imath\hat{H}t/\hbar)}.
\]



The time dependence is then in the operator itself, and this yields in turn the
following equation of motion

\[
\imath \hbar\frac{\partial }{\partial t}\hat{O}_H(t) = \exp{(\imath\hat{H}t/\hbar)}\left[\hat{O}_H\hat{H}-\hat{H}\hat{O}_H\right]\exp{(-\imath\hat{H}t/\hbar)}=\left[\hat{O}_H(t),\hat{H}\right].
\]

We note that an operator in the Heisenberg picture can be related to the corresponding
operator in the interaction picture as 

\[
\hat{O}_H(t)=
\exp{(\imath\hat{H}t/\hbar)}\hat{O}_S\exp{(-\imath\hat{H}t/\hbar)}=
\]


\[
\exp{(\imath\hat{H}_It/\hbar)}\exp{(-\imath\hat{H}_0t/\hbar)}\hat{O}_I\exp{(\imath\hat{H}_0t/\hbar)}\exp{(-\imath\hat{H}_It/\hbar)}.
\]

With our definition of the time evolution operator we see that

\[
\hat{O}_H(t)=\hat{U}(0,t)\hat{O}_I\hat{U}(t,0),
\]

which in turn implies that $\hat{O}_S=\hat{O}_I(0)=\hat{O}_H(0)$, all operators are equal at $t=0$. The wave function in the Heisenberg formalism is 
related to the other pictures as 

\[
|\Psi_H\rangle=|\Psi_S(0)\rangle=|\Psi_I(0)\rangle,
\]

since the wave function in the Heisenberg picture is time independent. 
We can relate this wave function to that a given time $t$ via the time evolution operator as

\[
|\Psi_H\rangle=\hat{U}(0,t)|\Psi_I(t)\rangle.
\]


\subsection{Adiabatic switching}


We assume that the interaction term is switched on gradually. Our wave function at time $t=-\infty$ and $t=\infty$ is supposed to represent a non-interacting system
given by the solution to the unperturbed part of our Hamiltonian $\hat{H}_0$.
We assume the ground state is given by $|\Phi_0\rangle$, which could be a Slater determinant.
We define our Hamiltonian as

\[
\hat{H}=\hat{H}_0+\exp{(-\varepsilon t/\hbar)}\hat{H}_I,
\]

where $\varepsilon$ is a small number. The way we write the Hamiltonian 
and its interaction term is meant to simulate the switching of the interaction.



The time evolution of the wave function in the interaction picture is then

\[
|\Psi_I(t) \rangle = \hat{U}_{\varepsilon}(t,t_0)|\Psi_I(t_0)\rangle,
\]

with 

\[
\hat{U}_{\varepsilon}(t,t_0)=\sum_{n=0}^{\infty}\left(\frac{-\imath}{\hbar}\right)^n\frac{1}{n!}
\int_{t_0}^t dt_1\dots \int_{t_0}^t dt_N
\]


\[
\times \exp{(-\varepsilon(t_1+\dots+t_n)/\hbar)}\hat{T}\left[\hat{H}_I(t_1)\dots\hat{H}_I(t_n)\right]
\]



\subsection{Initial state preparation}

In the limit $t_0\rightarrow -\infty$, the solution ot Schrödinger's equation is
$|\Phi_0\rangle$, and the eigenenergies are given by 

\[
\hat{H}_0|\Phi_0\rangle=W_0|\Phi_0\rangle,
\]

meaning that 

\[
|\Psi_S(t_0)\rangle = \exp{(-\imath W_0t_0/\hbar)}|\Phi_0\rangle,
\]

with the corresponding interaction picture wave function given by

\[
|\Psi_I(t_0)\rangle = \exp{(\imath \hat{H}_0t_0/\hbar)}|\Psi_S(t_0)\rangle=|\Phi_0\rangle.
\]



The solution becomes time independent in the limit $t_0\rightarrow -\infty$.
The same conclusion can be reached by looking at 

\[
\imath \hbar\frac{\partial }{\partial t}|\Psi_I(t)\rangle =
\exp{(\varepsilon |t|/\hbar)}\hat{H}_I|\Psi_I(t)\rangle 
\]

and taking the limit $t\rightarrow -\infty$.
We can rewrite the equation for the wave function at a time $t=0$ as

\[
|\Psi_I(0) \rangle = \hat{U}_{\varepsilon}(0,-\infty)|\Phi_0\rangle.
\]


\subsection{Specific realizations and famous gates}

Nuclear magnetic resonance (NMR) quantum computing is one of the several
proposed approaches for constructing a quantum computer. It uses the
spin states of nuclei within molecules as qubits. The quantum states
are probed through the nuclear magnetic resonances, allowing the
system to be implemented as a variation of nuclear magnetic resonance
spectroscopy. NMR differs from other implementations of quantum
computers in that it uses an ensemble of systems, in this case
molecules, rather than a single pure state.

In order to understand in terms of a given Hamiltonian how the
different gates arise, we consider now the Hamiltonian of a nuclear
spin in a magnetic field. Since the spin provides provides a magnetic
dipole moment, a nucleus with a spin will interact with the magnetic
field. The Haniltonian of a nucleus with spin interacting with a
magnetic field $\bm{B}$ is


\[
H = -\bm{\mu}\bm{B},
\]

with $\bm{\mu}=\gamma\bm{S}$, $\gamma$ being the so-called gyromagnetic ratio and $\bm{S}$ the spin.



It is common to let the spin interact with a constant magnetic field
along the $z$-axis. This gives an effecitve Hamiltonian


\[
H_z = -\frac{\hbar\omega_L}{2}\sigma_z,
\]


where $\omega_L$ is the so-called Larmor precession frequency. This
quantity includes also the constant magnetif field along the
$z$-axis. For all practical purposes it suffices for us to have an
expression of the Hamiltonian in terms of the Pauli-Z matrix.


Suppose that our initial one qubit state (for example a spin-$1/2$
nucleus for NMR studies) points along some arbitrary axis. As
discussed during our second week, a point on the Bloch sphere can be
represented as at time $t=0$

\[
\vert \psi(t=0) \rangle = \vert \psi(0) \rangle=\cos{(\frac{\theta}{2})}\vert 0\rangle +\exp{\imath\phi}\sin{(\frac{\theta}{2})}\vert 1\rangle.
\]



Since the hamiltonian is time-independent, the state $\vert \psi(0)
\rangle$, our system will evolve according to the unitary transformation 

\[
\vert \psi(t) \rangle = U(t)\vert \psi(0) \rangle=\exp{\imath\omega_L t\sigma_z/2}\vert \psi(0) \rangle.
\]

Inserting the Bloch sphere ansatz we have then

\[
\vert \psi(t) \rangle=\exp{\imath\omega_L t\sigma_z/2}\cos{(\frac{\theta}{2})}\vert 0\rangle +\exp{\imath\omega_L t\sigma_z/2}\exp{\imath\phi}\sin{(\frac{\theta}{2})}\vert 1\rangle.
\]


The specific hamiltonian we have chosen here serves to exemplify how can represent physical operations in terms of specific gates, here a one-qubit gate more notes to come.

Assume we have a given operator $\bm{A}$ acting on a  vector space $\vert a\rangle$ with eigenvalues $a$ 

\[
\exp{\bm{A}}\vert a\rangle=\sum_{n=0}^{\infty} \frac{1}{n!}\bm{A}^n\vert a\rangle=\sum_{n=0}^{\infty} \frac{a^n}{n!}\vert a\rangle=\exp{a}\vert a\rangle.
\]


Using this result, we obtain

\[
\vert \psi(t) \rangle=\exp{\imath\omega_L t/2}\cos{(\frac{\theta}{2})}\vert 0\rangle +\exp{-\imath\omega_L t/2}\exp{\imath\phi}\sin{(\frac{\theta}{2})}\vert 1\rangle.
\]

% material to be added

