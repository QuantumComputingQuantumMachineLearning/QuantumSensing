\chapter{Linear algebra for quantum mechanics}

\subsection{Notations and definitions}

Vectors, matrices and higher-order tensors are always boldfaced, with vectors
given by lower case letter letters and matrices and higher-order tensors given by upper case letters.

Unless otherwise stated, the elements $v_i$ of a vector $\bm{v}$ are assumed to be real. That is a vector of length $n$ is defined as
$\bm{x}\in \mathbb{R}^{n}$ and if we have a complex vector we have $\bm{x}\in \mathbb{C}^{n}$.

For a matrix of dimension $n\times n$ we have 
$\bm{A}\in \mathbb{R}^{n\times n}$ and the first matrix element starts with row element (row-wise ordering) zero and column element zero.


\paragraph{Some  mathematical notations}
\begin{enumerate}
\item For all/any  $\forall$

\item Implies $\implies$

\item Equivalent $\equiv$

\item Real variable $\mathbb{R}$

\item Integer variable $\mathbb{I}$

\item Complex  variable $\mathbb{C}$
\end{enumerate}



\paragraph{Vectors}

We start by defining a vector $\bm{x}$  with $n$ components, with $x_0$ as our first element, as

\[
\bm{x} = \begin{bmatrix} x_0\\ x_1 \\ x_2 \\ \dots \\ \dots \\ x_{n-1} \end{bmatrix}.
\]
and its transpose 
\[
\bm{x}^{T} = \begin{bmatrix} x_0 & x_1 & x_2 & \dots & \dots & x_{n-1} \end{bmatrix},
\]
In case we have a complex vector we define the hermitian conjugate
\[
\bm{x}^{\dagger} = \begin{bmatrix} x_0^* & x_1^* & x_2^* & \dots & \dots & x_{n-1}^* \end{bmatrix},
\]


\paragraph{Inner products of vectors}

With a given vector $\bm{x}$, we define the inner product as
\[
\bm{x}^T \bm{x} = \sum_{i=0}^{n-1} x_ix_i=x_0^2+x_1^2+\dots + x_{n-1}^2,
\]
or in case of a complex vector
\[
\bm{x}^{\dagger} \bm{x} = \sum_{i=0}^{n-1} x_i^*x_i=\vert x_0\vert^2+\vert x_1\vert^2+\dots + \vert x_{n-1}\vert^2,
\]


\paragraph{Hermitian conjugate}

The hermitian conjugate of a matrix is obtained by taking the complex
conjugate of each element and then taking the transpose of the
resulting matrix. Often we will just say the transpose or just the
conjugate, it should be clear from the context that we will mainly
deal with hermitian quantities and our matrices will in most cases be square matrices.

Unitarity, as we will see below, plays also a central role in this course.


\paragraph{Outer products}

In addition to inner products between vectors/states, the outer
product plays a central role in many applications. It is
defined as
\[
\bm{x}\bm{y}^T = \begin{bmatrix}
               x_0y_0 & x_0y_1 & x_0y_2 & \dots & \dots & x_0y_{n-2} & x_0y_{n-1} \\
	       x_1y_0 & x_1y_1 & x_1y_2 & \dots & \dots & x_1y_{n-2} & x_1y_{n-1} \\
	       x_2y_0 & x_2y_1 & x_2y_2 & \dots & \dots & x_2y_{n-2} & x_2y_{n-1} \\	       
               \dots &   \dots   & \dots  & \dots & \dots & \dots & \dots \\
               \dots &   \dots   & \dots  & \dots & \dots & \dots & \dots \\	       
	       x_{n-2}y_0 & x_{n-2}y_1 & x_{n-2}y_2 & \dots & \dots & x_{n-2}y_{n-2} & x_{n-2}y_{n-1} \\
	       x_{n-1}y_0 & x_{n-1}y_1 & x_{n-1}y_2 & \dots & \dots & x_{n-1}y_{n-2} & x_{n-1}y_{n-1} \end{bmatrix}	       
\]
The latter defines also our basic matrix layout.


\paragraph{Basic Matrix Features}

A general $n\times n$ matrix is given by 
\[
 \bm{A} =
\begin{bmatrix}
               a_{00} & a_{01} & a_{02} & \dots & \dots & a_{0n-2} & a_{0n-1} \\
               a_{10} & a_{11} & a_{12} & \dots & \dots & a_{1n-2} & a_{1n-1} \\
               \dots &   \dots   & \dots  & \dots & \dots & \dots & \dots \\
               \dots &   \dots   & \dots  & \dots & \dots & \dots & \dots \\	       
               a_{n-20} & a_{n-21} & a_{n-22} & \dots & \dots & a_{n-2n-2} & a_{n-2n-1} \\
               a_{n-10} & a_{n-11} & a_{n-12} & \dots & \dots & a_{n-1n-2} & a_{n-1n-1} \end{bmatrix},	       
\]
or in terms of its column vectors $\bm{a}_i$ as
\[
 \bm{A} =
\begin{bmatrix}\bm{a}_{0} & \bm{a}_{1} & \bm{a}_{2} & \dots & \dots & \bm{a}_{n-2} & \bm{a}_{n-1}\end{bmatrix}.	       
\]
We can think of a matrix as a diagram of in general $n$ rowns and $m$ columns. In the example here we have a square matrix.


\paragraph{The inverse of a matrix}
\begin{block}{}
The inverse of a square matrix (if it exists) is defined by

\[
\bm{A}^{-1} \cdot \bm{A} = I,
\]
where $\bm{I}$ is the unit matrix.
\end{block}


\paragraph{Selected Matrix Features}


{\footnotesize
\begin{tabular}{ccc}
\hline
\multicolumn{1}{c}{ Relations } & \multicolumn{1}{c}{ Name } & \multicolumn{1}{c}{ matrix elements } \\
\hline
$\bm{A} = \bm{A}^{T}$                            & symmetric       & $a_{ij} = a_{ji}$                                                       \\
$\bm{A} = \left (\bm{A}^{T} \right )^{-1}$       & real orthogonal & $\sum_k a_{ik} a_{jk} = \sum_k a_{ki} a_{kj} = \delta_{ij}$             \\
$\bm{A} = \bm{A}^{ * }$                          & real matrix     & $a_{ij} = a_{ij}^{ * }$                                                 \\
$\bm{A} = \bm{A}^{\dagger}$                      & hermitian       & $a_{ij} = a_{ji}^{ * }$                                                 \\
$\bm{A} = \left (\bm{A}^{\dagger} \right )^{-1}$ & unitary         & $\sum_k a_{ik} a_{jk}^{ * } = \sum_k a_{ki}^{ * } a_{kj} = \delta_{ij}$ \\
\hline
\end{tabular}
}



\paragraph{Some famous Matrices}

\begin{itemize}
  \item Diagonal if $a_{ij}=0$ for $i\ne j$

  \item Upper triangular if $a_{ij}=0$ for $i > j$

  \item Lower triangular if $a_{ij}=0$ for $i < j$

  \item Upper Hessenberg if $a_{ij}=0$ for $i > j+1$

  \item Lower Hessenberg if $a_{ij}=0$ for $i < j+1$

  \item Tridiagonal if $a_{ij}=0$ for $|i -j| > 1$

  \item Lower banded with bandwidth $p$: $a_{ij}=0$ for $i > j+p$

  \item Upper banded with bandwidth $p$: $a_{ij}=0$ for $i < j+p$

  \item Banded, block upper triangular, block lower triangular....
\end{itemize}



\paragraph{Matrix Features}

\begin{block}{Some equivalent statements for square matrices }
For an $n\times n$ matrix  $\bm{A}$ the following properties are all equivalent

\begin{itemize}
  \item If the inverse of $\bm{A}$ exists, $\bm{A}$ is nonsingular.

  \item The equation $\bm{Ax}=0$ implies $\bm{x}=0$.

  \item The rows of $\bm{A}$ form a basis of $\mathbb{C}^{n}$.

  \item The columns of $\bm{A}$ form a basis of $\mathbb{C}^{n}$.

  \item $\bm{A}$ is a product of elementary matrices. Can you name one example?

  \item $0$ is not an eigenvalue of $\bm{A}$.
\end{itemize}


\end{block}


\paragraph{Important Mathematical Operations}

The basic matrix operations that we will deal with are addition and subtraction

\[
\bm{A}= \bm{B}\pm\bm{C}  \Longrightarrow a_{ij} = b_{ij}\pm c_{ij},
\]
and scalar-matrix multiplication

\[
\bm{A}= \gamma\bm{B}  \Longrightarrow a_{ij} = \gamma b_{ij}.
\]


\paragraph{Vector-matrix and Matrix-matrix multiplication}

We have also vector-matrix multiplications 
\[
\bm{y}=\bm{Ax}   \Longrightarrow y_{i} = \sum_{j=0}^{n-1} a_{ij}x_j,
\]
and matrix-matrix multiplications

\[
\bm{A}=\bm{BC}   \Longrightarrow a_{ij} = \sum_{k=0}^{n-1} b_{ik}c_{kj},
\]
and transpositions of a matrix

\[
\bm{A}=\bm{B}^T   \Longrightarrow a_{ij} = b_{ji}.
\]


\paragraph{Important Mathematical Operations}

Similarly, important vector operations that we will deal with are addition and subtraction

\[
\bm{x}= \bm{y}\pm\bm{z}  \Longrightarrow x_{i} = y_{i}\pm z_{i},
\]
scalar-vector multiplication

\[
\bm{x}= \gamma\bm{y}  \Longrightarrow x_{i} = \gamma y_{i},
\]


\paragraph{Other important mathematical operations}
and vector-vector multiplication (called Hadamard multiplication)
\[
\bm{x}=\bm{yz}   \Longrightarrow x_{i} = y_{i}z_i.
\]
Finally, as already metnioned, the inner or so-called dot product  resulting in a constant

\[
x=\bm{y}^T\bm{z}   \Longrightarrow x = \sum_{j=0}^{n-1} y_{j}z_{j},
\]
and the outer product, which yields a matrix,

\[
\bm{A}=  \bm{y}\bm{z}^T \Longrightarrow  a_{ij} = y_{i}z_{j},
\]


\paragraph{Defining basis states and quantum mechanical operators}

We extend now to quantum mechanics our definitions of vectors, matrices and more.

We start by defining a state vector $\bm{x}$ (meant to represent
various quantum mechanical degrees of freedom) with $n$ components as

\[
\bm{x} = \begin{bmatrix} x_0\\ x_1 \\ x_2 \\ \dots \\ \dots \\ x_{n-1} \end{bmatrix}.
\]


\paragraph{Dirac notation}

Throughout these notes we will use the so-called Dirac \textbf{bra-ket}
formalism and we will replace the above standard boldfaced notation
for a vector with

\[
\bm{x} = \vert x \rangle = \begin{bmatrix} x_0\\ x_1 \\ x_2 \\ \dots \\ \dots \\ x_{n-1} \end{bmatrix},
\]
and
\[
\bm{x}^{\dagger} = \langle x \vert = \begin{bmatrix} x_0^* & x_1^* & x_2^* & \dots & \dots & x_{n-1}^* \end{bmatrix},
\]


\paragraph{Inner product in Dirac notation}

With a given vector $\vert x \rangle$, we define the inner product as
\[
\langle x \vert x\rangle = \sum_{i=0}^{n-1} x_i^*x_i=\vert x_0\vert^2+\vert x_1\vert ^2+\dots + \vert x_{n-1}\vert^2. 
\]

For two arbitrary vectors $\vert x\rangle$ and $\vert y\rangle$ with the same lentgh, we have the
general expression
\[
\langle y \vert x\rangle = \sum_{i=0}^{n-1} y_i^*x_i=y_0^*x_0+y_1^*x_1+\dots + y_{n-1}^*x_{n-1}. 
\]


\paragraph{The inner product is a real number}

\begin{block}{}
Note well that the inner product $\langle x \vert x\rangle$ is always a real number while for a two different vectors $\langle y \vert x\rangle$ is in general not equal to
$\langle x \vert y\rangle$, as can be seen from the example in the next slide. 
\end{block}

\begin{block}{}
We note in bypassing that $\vert x\rangle^{\dagger}=\langle x \vert$,
$\langle x\vert^{\dagger}=\vert x\rangle$ and $(\vert
x\rangle^{\dagger})^{\dagger}=\vert x \rangle$.
\end{block}


\paragraph{Examples}

Let us assume that $\vert x \rangle$ is given by
\[
\vert x \rangle = \begin{bmatrix} 1-\imath \\ 2+\imath \end{bmatrix}.
\]
The inner product gives us
\[
\langle x\vert x \rangle = (1+\imath)(1-\imath)+(2-\imath)(2+\imath)=7,
\]
a real number.


\paragraph{Norm}
We can use the norm/inner product to normalize the vector $\vert x \rangle$ and obtain
\[
\vert x \rangle = \frac{1}{\sqrt{7}}\begin{bmatrix} 1-\imath \\ 2+\imath \end{bmatrix}.
\]

As another example, consider the two vectors
\[
\vert x \rangle = \begin{bmatrix} -1 \\ 2\imath \\ 1\end{bmatrix},
\]
and
\[
\vert y \rangle = \begin{bmatrix} 1 \\ 0\imath \\ \imath\end{bmatrix}.
\]
We see that the inner products $\langle x\vert y \rangle = -1+\imath$, which is not the same as
$\langle y\vert x \rangle = -1-\imath$. This leads to the important rule
\[
\langle x\vert y\rangle^* = \langle y \vert x\rangle. 
\]


\paragraph{Outer products}

In addition to inner products between vectors/states, the outer
product plays a central role in all of quantum mechanics. It is
defined as
\[
\vert x\rangle \langle y \vert = \begin{bmatrix}
               x_0y_0^* & x_0y_1^* & x_0y_2^* & \dots & \dots & x_0y_{n-2}^* & x_0y_{n-1}^* \\
	       x_1y_0^* & x_1y_1^* & x_1y_2^* & \dots & \dots & x_1y_{n-2}^* & x_1y_{n-1}^* \\
	       x_2y_0^* & x_2y_1^* & x_2y_2^* & \dots & \dots & x_2y_{n-2}^* & x_2y_{n-1}^* \\	       
               \dots &   \dots   & \dots  & \dots & \dots & \dots & \dots \\
               \dots &   \dots   & \dots  & \dots & \dots & \dots & \dots \\	       
	       x_{n-2}y_0^* & x_{n-2}y_1^* & x_{n-2}y_2^* & \dots & \dots & x_{n-2}y_{n-2}^* & x_{n-2}y_{n-1}^* \\
	       x_{n-1}y_0^* & x_{n-1}y_1^* & x_{n-1}y_2^* & \dots & \dots & x_{n-1}y_{n-2}^* & x_{n-1}y_{n-1}^* \end{bmatrix}	       
\]


\paragraph{Different operators and gates}

In quantum computing, the so-called Pauli matrices, and other simple
$2\times 2$ matrices, play an important role, ranging from the setup
of quantum gates to a rewrite of creation and annihilation operators
and other quantum mechanical operators. Let us start with the familiar
Pauli matrices and remind ourselves of some of their basic properties.

The Pauli matrices are defined as
\[
\sigma_x = \begin{bmatrix} 0 & 1 \\ 1 & 0 \end{bmatrix},
\]
\[
\sigma_y = \begin{bmatrix} 0 & -\imath \\ \imath & 0 \end{bmatrix},
\]
and
\[
\sigma_z = \begin{bmatrix} 1 & 0 \\ 0 & -1 \end{bmatrix}.
\]


\paragraph{Properties of Pauli matrices}

It is easy to show that the matrices obey the properties (being involutory)
\[
\sigma_x\sigma_x = \sigma_y\sigma_y=\sigma_z\sigma_z = I=\begin{bmatrix} 1 & 0 \\ 0 & 1\end{bmatrix},
\]

that is their products with themselves result in the identity matrix
$\bm{I}$.  Furthermore, the Pauli matrices are unitary matrices
meaning that their inverses are equal to their hermitian conjugated
matrices. The determinants of the Pauli matrices are all equal to $-1$,
as can be easily verified.


\paragraph{Commutation relations}

The Pauli matrices obey also the following commutation rules
\[
\left[\sigma_x,\sigma_y\right] = 2\imath \sigma_z.
\]

Before we proceed with other matrices and how they can be used to
operate on various quantum mechanical states, let us try to define
various basis sets and their pertinent notations. We will often refer
to these basis states as our computational basis.


\paragraph{Definition of Computational basis states}

% to do: make figures with examples of basis states, hydrogen like systems, harmonic oscillator
Assume we have a two-level system where the two states are represented
by the state vectors $\vert \phi_0\rangle$ and $\vert \phi_1\rangle$,
respectively. These states could represent selected or effective
degrees of freedom for either a single particle (fermion or boson) or
they could represent effective many-body degrees of freedon. In actual
realizations of quantum computing we search often for candidate
systems where we can use some low-lying states as computational basis
states. But we are not limited to quantum computing. When doing
many-body physics, due to the exploding degrees of freedom, we
normally search after effective ways by which we can reduce the
involved dimensionalities to a number of degrees of freedom we can
handle by a given many-body method.


\paragraph{Projection operators}

We will now relabel the above two states as two orthogonal and normalized basis (ONB) states 
\[
\vert \phi_0 \rangle = \vert 0 \rangle = \begin{bmatrix} 1 \\ 0 \end{bmatrix},
\]
and 
\[
\vert \phi_1 \rangle = \vert 1 \rangle = \begin{bmatrix} 0 \\ 1 \end{bmatrix}.
\]


\paragraph{Identity and projection operators}

It is straight forward to see that $\langle 1 \vert 0\rangle=0$. With these two states we can define the define the identity operator $\bm{I}$ as the sum of the outer products of these two states, namely
\[
\bm{I} = \sum_{i=0}^{i=1}\vert i\rangle \langle i\vert = \begin{bmatrix} 1 & 0 \\ 0 & 0 \end{bmatrix} +\begin{bmatrix} 0 & 0 \\ 0 & 1 \end{bmatrix}=\begin{bmatrix} 1 & 0 \\ 0 & 1 \end{bmatrix}.
\]
We can further define the projection operators
\[
\bm{P} = \vert 0\rangle \langle 0\vert = \begin{bmatrix} 1 & 0 \\ 0 & 0 \end{bmatrix},
\]
and 
\[
\bm{Q} = \vert 1\rangle \langle 1\vert = \begin{bmatrix} 0 & 0 \\ 0 & 1 \end{bmatrix}.
\]


\paragraph{Idempotent operators}

We note that $P^2=P$, $Q^2=Q$ (the operators are idempotent) and that
their determinants are zero, meaning in turn that we cannot use these
operators for unitary/orthogonal transformations. However, they play
important roles in defining effective Hilbert spaces for many-body
studies. Finally, before proceeding we note also that the two matrices
commute and we have $\bm{P}\bm{Q}=0$ and $\left[ \bm{P},\bm{Q}\right]=0$.


\paragraph{Superposition and more}

Using the properties of ONBs we can expand a new state in terms of the
above states. These states could also form  a basis which is an
eigenbasis of a selected Hamiltonian (more of this below).

We define now a new state which is a linear expansion in terms of
these computational basis states

\[
\vert \psi \rangle = \alpha \vert 0 \rangle + \beta\vert 1 \rangle,
\]

where the coefficients $\alpha = \langle 0 \vert \psi \rangle$ and
$\beta =\langle 1 \vert \psi\rangle$ reresent the overlaps between the
computational basis states and the state $\vert \psi\rangle$. In quantum speech, we say the state is in a superposition of the states $\vert 0\rangle$ and $\vert 1\rangle$.


\paragraph{Inner products}
Computing the inner product of $\vert \psi \rangle$ we obtain
\[
\langle \psi \vert \psi \rangle = \vert \alpha \vert ^2\langle 0\vert 0\rangle + \vert \beta \vert ^2\langle 1\vert 1\rangle = \vert \alpha \vert ^2 + \vert \beta \vert ^2 = 1,
\]

since the new basis, which is defined in terms of a a unitary/orthogonal
transformation, preserves the orthogonality and norm of the original
computational basis $\vert 0\rangle$ and $\vert 1\rangle$. To see
this, consider the unitary transformation (show derivation of
preserving orthogonality).


\paragraph{Acting with projection operators}

If we now act with the projection operators $\bm{P}$ and $\bm{Q}$ on
the state $\vert \psi\rangle$ we get

\[
\bm{P}\vert \psi \rangle = \vert 0 \rangle\langle 0\vert (\alpha \vert 0 \rangle + \beta\vert 1 \rangle)=\alpha \vert 0\rangle,
\]

that is we \textbf{project} out the $\vert 0\rangle$ component of the state
$\vert \psi\rangle$ with the coefficient $\alpha$ while $\bm{Q}$
projects out the $\vert 1\rangle$ component with coefficient $\beta$
as seen from

\[
\bm{Q}\vert \psi \rangle = \vert 1 \rangle\langle 1\vert (\alpha \vert 0 \rangle + \beta\vert 1 \rangle)=\beta \vert 1\rangle.
\]

The above results can easily be derived by multiplying the pertinent
matrices with the vectors $\vert 0\rangle$ and $\vert 1\rangle$,
respectively.


\paragraph{Density matrix}

Using the above linear expansion we can now define the density matrix of the state $\vert \psi\rangle$ as the outer product
\[
\bm{\rho}=\vert \psi \rangle\langle \psi \vert = \alpha\alpha^* \vert 0 \rangle\langle 0\vert+\alpha\beta^* \vert 0 \rangle\langle 1\vert+\beta\alpha^* \vert 1 \rangle\langle 0\vert+\beta\beta^* \vert 1 \rangle\langle 1\vert,
\]
which leads to
\[
\bm{\rho}=\begin{bmatrix} \alpha\alpha^* & \alpha\beta^*\\ \beta\alpha^* & \beta\beta^*\end{bmatrix}.
\]

Finally, we note that the trace of the density matrix is simply given by unity
\[
\mathrm{tr}\bm{\rho}=\alpha\alpha^* +\beta\beta^*=1.
\]


\paragraph{Other important matrices}

We present other operators (as matrices) which play an important role in quantum computing, the so-called Hadamard matrix (or gate as we will use later)
\[
\bm{H}=\frac{1}{\sqrt{2}}\begin{bmatrix} 1 & 1 \\ 1 & -1\end{bmatrix}.
\]

The action of the operator $\bm{H}$ on a computational basis state like $\vert 0\rangle$ gives
\[
\bm{H}\vert 0 \rangle = \frac{1}{\sqrt{2}}\begin{bmatrix} 1 & 1 \\ 1 & -1\end{bmatrix}\begin{bmatrix} 1 \\ 0\end{bmatrix}=\frac{1}{\sqrt{2}}(\vert 0\rangle + \vert 1\rangle),
\]
and 

\[
\bm{H}\vert 1 \rangle = \frac{1}{\sqrt{2}}\begin{bmatrix} 1 & 1 \\ 1 & -1\end{bmatrix}\begin{bmatrix} 0 \\ 1\end{bmatrix}=\frac{1}{\sqrt{2}}(\vert 0\rangle - \vert 1\rangle),
\]
that is we create a superposition of the states $\vert 0\rangle$ and $\vert 1\rangle$.


\paragraph{Phase matrix}
Another famous operation is the phase matrix given by
\[
\bm{S} = \begin{bmatrix} 1 & 0 \\ 0 & \imath\end{bmatrix}.
\]


\paragraph{Tensor products}

Consider now two vectors with length $n=2$, with elements

\[
\vert x \rangle = \begin{bmatrix} x_0 \\ x_1 \end{bmatrix}, 
\]
and
\[
\vert y \rangle = \begin{bmatrix} y_0 \\ y_1 \end{bmatrix}. 
\]
The tensor product of these two vectors is defined as
\[
\vert x \rangle \otimes \vert y \rangle = \vert xy \rangle  = \begin{bmatrix} x_0y_0 \\ x_0y_1 \\ x_1y_0 \\ x_1y_1 \end{bmatrix}, 
\]
which is now a vector of length $4$.


\paragraph{Examples of tensor products}
If we now go back to our original one-qubit basis states, we can form teh following tensor products
\[
\vert 0 \rangle \otimes \vert 0 \rangle = \begin{bmatrix} 1 \\ 0\end{bmatrix} \otimes \begin{bmatrix} 1 \\ 0\end{bmatrix} =\begin{bmatrix} 1 \\ 0 \\ 0 \\ 0 \end{bmatrix}=\vert 00 \rangle, 
\]
\[
\vert 0 \rangle \otimes \vert 1 \rangle = \begin{bmatrix} 1 \\ 0\end{bmatrix} \otimes \begin{bmatrix} 0 \\ 1\end{bmatrix} =\begin{bmatrix} 0 \\ 1 \\ 0 \\ 0 \end{bmatrix}=\vert 01 \rangle.
\]


\paragraph{More states}

\[
\vert 1 \rangle \otimes \vert 0 \rangle = \begin{bmatrix} 0 \\ 1\end{bmatrix} \otimes \begin{bmatrix} 1 \\ 0\end{bmatrix} =\begin{bmatrix} 0 \\ 0 \\ 1 \\ 0 \end{bmatrix}=\vert 10 \rangle, 
\]
and finally
\[
\vert 1 \rangle \otimes \vert 1 \rangle = \begin{bmatrix} 0 \\ 1\end{bmatrix} \otimes \begin{bmatrix} 0 \\ 1\end{bmatrix} =\begin{bmatrix} 0 \\ 0 \\ 0 \\ 1 \end{bmatrix}=\vert 11 \rangle. 
\]


\paragraph{Three qubits}
We have now four different states and we could try to make a new list
by relabeling the states as follows $\vert 00 \rangle =\vert 0
\rangle$, $\vert 01 \rangle =\vert 1 \rangle$, $\vert 10 \rangle
=\vert 2 \rangle$, $\vert 11 \rangle =\vert 3 \rangle$.

In similar ways we can define the tensor product of three qubits (or single-particle states) as
\[
\vert 0 \rangle \otimes \vert 0 \rangle \otimes \vert 0 \rangle = \begin{bmatrix} 1 \\ 0\end{bmatrix} \otimes \begin{bmatrix} 1 \\ 0\end{bmatrix} \otimes \begin{bmatrix} 1 \\ 0\end{bmatrix}=\begin{bmatrix} 1 \\ 0 \\ 0 \\ 0 \\ 0 \\0 \\ 0 \\ 0\end{bmatrix}=\vert 000 \rangle, 
\]
which is a new vector of length eight.


\paragraph{Generalizing}

We note that with a single-particle basis given the states $\vert
0\rangle$ and $\vert 1\rangle$ we can, with $N$ particles construct
$2^N$ different states.  This is something we can generalize to
\begin{itemize}
\item discuss ways of labeling states

\item how to write a code which does it
\end{itemize}



\paragraph{Tensor products of matrices}
The tensor product of two $2\times 2$ matrices $\bm{A}$ and $\bm{B}$ is given by

\[
\bm{A} \times \bm{B} = \begin{bmatrix} a_{00} & a_{01} \\ a_{10} & a_{11} \end{bmatrix} \otimes \begin{bmatrix} b_{00} & b_{01} \\ b_{10} & b_{11} \end{bmatrix} =
\begin{bmatrix} a_{00} b_{00} &  a_{00}b_{01} & a_{01} b_{00} & a_{01}b_{01} \\
                a_{00} b_{10} &  a_{00}b_{11} & a_{01} b_{10} & a_{01}b_{11} \\
                a_{10} b_{00} &  a_{10}b_{01} & a_{11} b_{00} & a_{11}b_{01} \\
                a_{10} b_{10} &  a_{10}b_{11} & a_{11} b_{10} & a_{11}b_{11} \end{bmatrix}
\]


\paragraph{Measurements}

The probability of a measurement on a quantum system giving a certain
result is determined by the weight of the relevant basis state in the
state vector. After the measurement, the system is in a state that
corresponds to the result of the measurement. The operators and
gates discussed below are examples of operations we can perform on
specific states.

We  consider the state
\[
\vert \psi\rangle = \alpha \vert 0 \rangle +\beta \vert 1 \rangle
\]


\paragraph{Definitions of measurements}

\begin{enumerate}
\item A measurement can yield only one of the above states, either $\vert 0\rangle$ or $\vert 1\rangle$.

\item The probability of a measurement resulting in $\vert 0\rangle$ is $\alpha^*\alpha = \vert \alpha \vert^2$.

\item The probability of a measurement resulting in $\vert 1\rangle$ is $\beta^*\beta = \vert \beta \vert^2$.

\item And we note that the sum of the outcomes gives $\alpha^*\alpha+\beta^*\beta=1$ since the two states are normalized.
\end{enumerate}


After the measurement, the state of the system is the state associated with the result of the measurement.

We have already encountered the projection operators $P$ and $Q$. Let
us now look at other types of operations we can make on qubit states.


\paragraph{Different operators and gates}

In quantum computing, the so-called Pauli matrices, and other simple
$2\times 2$ matrices, play an important role, ranging from the setup
of quantum gates to a rewrite of creation and annihilation operators
and other quantum mechanical operators. Let us start with the familiar
Pauli matrices and remind ourselves of some of their basic properties.

Assume we operate with $\sigma_x$ on our basis state $\vert 0 \rangle$. This gives
\[
\begin{bmatrix} 0 & 1 \\ 1 & 0 \end{bmatrix}\begin{bmatrix} 1 \\ 0 \end{bmatrix}=\begin{bmatrix} 0  \\ 1  \end{bmatrix},
\]
that is we switch from $\vert 0\rangle$ to $\vert 1\rangle$ (often called a spin flip operation) and similary we have
\[
\begin{bmatrix} 0 & 1 \\ 1 & 0 \end{bmatrix}\begin{bmatrix} 0 \\ 1 \end{bmatrix}=\begin{bmatrix} 1  \\ 0  \end{bmatrix}.
\]


\paragraph{More on Pauli matrices}

This matrix plays an important role in quantum computing since we can
link this with the classical \textbf{NOT} operation.  If we send in bit 0,
the \textbf{NOT} gate outputs bit 1 and vice versa. We can use the $\sigma_x$
matrix to implement the quantum mechanical equivalent of a classical
\textbf{NOT} gate. If we input what we could represent as bit 0 in terms of
the basis state $\vert 0\rangle$, operating on this state results in
the state $\vert 1\rangle$, which we in turn can interpret as the
classical bit 1.


\paragraph{Linear superposition}
If we have a linear superposition of these states we obtain
\[
\begin{bmatrix}0 & 1 \\ 1 & 0 \end{bmatrix}\begin{bmatrix}\alpha \\ \beta \end{bmatrix}=\begin{bmatrix}\beta \\ \alpha \end{bmatrix}.
\]

The $\sigma_y$ matrix introduces an imaginary sign, which we will later encounter in terms of so-called phase-shift operations.


\paragraph{The $\sigma_z$ matrix}
The $\sigma_z$ matrix has the following effect
\[
\begin{bmatrix} 1 & 0 \\ 0 & -1 \end{bmatrix}\begin{bmatrix} 1 \\ 0 \end{bmatrix}=\begin{bmatrix} 1  \\ 0  \end{bmatrix},
\]
and 
\[
\begin{bmatrix} 1 & 0 \\ 0 & -1 \end{bmatrix}\begin{bmatrix} 0 \\ 1 \end{bmatrix}=\begin{bmatrix} 0  \\ -1  \end{bmatrix},
\]
which we will also link with a specific phase-shift.


\paragraph{Unitarity}

The matrices we introduced here are so-called unitary matrices. This
is an important element in quantum mechanics since the evolution of a
closed quantum system is described by operations involving unitary
operations only.

We have defined a new state $\vert \psi_p\rangle$ as a linear expansion in terms of an orthogonal and normalized basis (our computational basis) $\phi_{\lambda}$
\begin{equation}
\vert \psi_i\rangle = \sum_{j} u_{ij}\vert \phi_{j}\rangle.
\end{equation}


\paragraph{Hamiltonians and basis functions}

It is normal to choose a basis defined as the eigenfunctions of parts
of the full Hamiltonian. The typical situation consists of the
solutions of the one-body part of the Hamiltonian, that is we have

\[
\hat{h}_0\vert \phi_{i}\rangle=\epsilon_{i}\vert \phi_{i}\rangle.
\]

This is normally referred to as a single-particle basis $\vert\phi_{i}(\mathbf{r})\rangle$,
defined by the quantum numbers $i$ and $\mathbf{r}$.


\paragraph{Unitary transformations}
A unitary transformation is important since it keeps the orthogonality.
To see this consider first a basis of vectors $\mathbf{v}_i$,
\[
\mathbf{v}_i = \begin{bmatrix} v_{i1} \\ \dots \\ \dots \\v_{in} \end{bmatrix}
\]
We assume that the basis is orthogonal, that is 
\[
\mathbf{v}_j^T\mathbf{v}_i = \delta_{ij}.
\]
An orthogonal or unitary transformation
\[
\mathbf{w}_i=\mathbf{U}\mathbf{v}_i,
\]
preserves the dot product and orthogonality since
\[
\mathbf{w}_j^T\mathbf{w}_i=(\mathbf{U}\mathbf{v}_j)^T\mathbf{U}\mathbf{v}_i=\mathbf{v}_j^T\mathbf{U}^T\mathbf{U}\mathbf{v}_i= \mathbf{v}_j^T\mathbf{v}_i = \delta_{ij}.
\]


\paragraph{Orthogonality preserved}

This means that if the coefficients $u_{p\lambda}$ belong to a unitary
or orthogonal transformation (using the Dirac bra-ket notation)

\[
\vert \psi_i\rangle = \sum_{j} u_{ij}\vert \phi_{j}\rangle.
\]

orthogonality is preserved.

This propertry is extremely useful when we build up a basis of
many-body determinant based states.

Note also that although a basis $\left\{\vert \phi_i \rangle\right\}$ contains an infinity of states, for practical calculations we have always to make some truncations.


\paragraph{Example}

Assume we have two one-qubit states represented by
\[
\vert \psi \rangle = \alpha \vert 0 \rangle + \beta \vert 1\rangle=\begin{bmatrix}\alpha \\ \beta \end{bmatrix},
\]
and
\[
\vert \phi \rangle = \gamma \vert 0 \rangle + \delta \vert 1\rangle=\begin{bmatrix}\gamma \\ \delta \end{bmatrix}.
\]

We assume that the state $\vert \phi \rangle$ is obtained through a
unitary transformation of $\vert \psi \rangle$ through a matrix
$\bm{U}$ with its hermitian conjugate $\bm{U}^{\dagger}$ with matrix
elements $u_{ij}^{\dagger}=u_{ji}^*$ and
$\bm{I}=\bm{U}\bm{U}^{\dagger}=\bm{U}^{\dagger}\bm{U}$.


\paragraph{Inverse of unitary matrices}

Note that this means that the hermitian conjugate of a unitary matrix
is equal to its inverse. This has important consequences for what is
called reversibility. We say quantum mechanics is a theory which is
reversible with a probabilistic determinism. Classical mechanics on
the other is reversible in a deterministic way, that is, knowing all
initial conditions we can in principle determine the future motion of
an object which obey the laws of motion of classical mechanics.

We have then
\[
\begin{bmatrix}\gamma \\ \delta \end{bmatrix}=\begin{bmatrix}u_{00} & u_{01} \\ u_{10} & u_{11} \end{bmatrix}\begin{bmatrix}\alpha \\ \beta \end{bmatrix}.
\]


\paragraph{New basis is also orthogonal}

Since our original basis $\vert \psi\rangle$ is orthogonal and normalized with $\vert\alpha\vert^2+\vert\beta\vert^2=1$, the new basis is also orthogonal and normalized, as we can see below here.

Since the inverse of a hermitian matrix is equal to its hermitian
conjugate/adjoint), unitary transformations are always reversible.

Why are only unitary transformations allowed? The key lies in the way the inner product tranforms.

To see this we rewrite the new basis from the previous example in its two components as
\[
\vert \phi\rangle_i=\sum_{j}u_{ij}\vert \psi\rangle_j,
\]
or in terms of a matrix-vector notatio we have
\[
\vert \phi\rangle=\bm{U}\vert \psi\rangle,
\]


\paragraph{More on orthogonality}

We have already assumed that $\langle \psi \vert \psi \rangle = \vert\alpha\vert^2+\vert\beta\vert^2=1$.

We have that 
\[
\langle \phi\vert_i=\sum_{j}u_{ij}^*\langle \psi\vert_j,
\]
or in terms of a matrix-vector notation we have
\[
\langle \phi\vert=\langle \psi\vert\bm{U}^{\dagger}.
\]

Note that the two vectors are row vectors now.

If we stay with this notation we have

\[
\langle \phi\vert\phi\rangle = \langle \psi \bm{U}^{\dagger}\bm{U}\vert \psi\rangle = \langle \psi\vert \psi\rangle=1!
\]

Unitary transformations are rotations in state space which preserve the
length (the square root of the inner product) of the state vector.





\paragraph{Basic properties of hermitian operators}

The operators we typically encounter in quantum mechanical studies are
\begin{enumerate}
\item Hermitian (self-adjoint) meaning that for example the elements of a Hermitian matrix $\bm{U}$ obey $u_{ij}=u_{ji}^*$.

\item Unitary $\bm{U}\bm{U}^{\dagger}=\bm{U}^{\dagger}\bm{U}=\bm{I}$, where $\bm{I}$ is the unit matrix

\item The oparator $\bm{U}$ and its self-adjoint commute (often labeled as normal operators), that is  $[\bm{U},\bm{U}^{\dagger}]=0$. An operator is \textbf{normal} if and only if it is diagonalizable. A Hermitian operator is normal.
\end{enumerate}


Unitary operators in a Hilbert space preserve the norm and orthogonality. If $\bm{U}$ is a unitary operator acting on a state $\vert \psi_j\rangle$, the action of

\[
\vert \phi_i\rangle=\bm{U}\vert \psi_j\rangle,
\]
preserves both the norm and orthogonality, that is $\langle \phi_i \vert \phi_j\rangle=\langle \psi_i \vert \psi_j\rangle=\delta_{ij}$, as discussed earlier.


\paragraph{The Pauli matrices again}

As example, consider the Pauli matrix $\sigma_x$. We have already seen that this matrix is a unitary matrix. Consider then an orthogonal and normalized basis $\vert 0\rangle^{\dagger} =\begin{bmatrix} 1 {\&} 0\end{bmatrix}$ and $\vert 1\rangle^{\dagger} =\begin{bmatrix} 0 {\&} 1\end{bmatrix}$ and a state which is a linear superposition of these two basis states

\[
\vert \psi_a\rangle=\alpha_0\vert 0\rangle +\alpha_1\vert 1\rangle.
\]

A new state $\vert \psi_b\rangle$ is given by
\[
\vert \psi_b\rangle=\sigma_x\vert \psi_a\rangle=\alpha_0\vert 1\rangle +\alpha_1\vert 0\rangle.
\]


\paragraph{Spectral Decomposition}

An important technicality which we will use in the discussion of
density matrices, entanglement, quantum entropies and other properties
is the so-called spectral decomposition of an operator.

Let $\vert \psi\rangle$ be a vector in a Hilbert space of dimension $n$ and a hermitian operator $\bm{A}$ defined in this
space. Assume $\vert \psi\rangle$ is an eigenvector of $\bm{A}$ with eigenvalue $\lambda$, that is

\[
\bm{A}\vert \psi\rangle = \lambda\vert \psi\rangle = \lambda \bm{I}\vert \psi \rangle,
\]
where we used $\bm{I}\vert \psi \rangle = 1 \vert \psi \rangle$.
Subtracting the right hand side from the left hand side gives
\[
\left[\bm{A}-\lambda \bm{I}\right]\vert \psi \rangle=0,
\]

which has a nontrivial solution only if the determinant
$\mathrm{det}(\bm{A}-\lambda\bm{I})=0$.


\paragraph{ONB again and again}

We define now an orthonormal basis $\vert i \rangle =\{\vert 0
\rangle, \vert 1\rangle, \dots, \vert n-1\rangle$ in the same Hilbert
space. We will assume that this basis is an eigenbasis of $\bm{A}$ with eigenvalues $\lambda_i$

We expand a new vector using this eigenbasis of $\bm{A}$
\[
\vert \psi \rangle = \sum_{i=0}^{n-1}\alpha_i\vert i\rangle,
\]
with the normalization condition $\sum_{i=0}^{n-1}\vert \alpha_i\vert^2$.
Acting with $\bm{A}$ on this new state results in

\[
\bm{A}\vert \psi \rangle = \sum_{i=0}^{n-1}\alpha_i\bm{A}\vert i\rangle=\sum_{i=0}^{n-1}\alpha_i\lambda_i\vert i\rangle.
\]


\paragraph{Projection operators}

If we then use that the outer product of any state with itself defines a projection operator we have the projection operators
\[
\bm{P}_{\psi} = \vert \psi\rangle\langle \psi\vert,
\]
and
\[
\bm{P}_{j} = \vert j\rangle\langle j\vert,
\]
we have that 
\[
\bm{P}_{j}\vert \psi\rangle=\vert j\rangle\langle j\vert\sum_{i=0}^{n-1}\alpha_i\vert i\rangle=\sum_{i=0}^{n-1}\alpha_i\vert j\rangle\langle j\vert i\rangle.
\]


\paragraph{Further manipulations}

This results in
\[
\bm{P}_{j}\vert \psi\rangle=\alpha_j\vert j\rangle,
\]
since $\langle j\vert i\rangle$.
With the last equation we can rewrite
\[
\bm{A}\vert \psi \rangle = \sum_{i=0}^{n-1}\alpha_i\lambda_i\vert i\rangle=\sum_{i=0}^{n-1}\lambda_i\bm{P}_i\vert \psi\rangle,
\]
from which we conclude that
\[
\bm{A}=\sum_{i=0}^{n-1}\lambda_i\bm{P}_i.
\]


\paragraph{Spectral decomposition}

This is the spectral decomposition of a hermitian and normal
operator. It is true for any state and it is independent of the
basis. The spectral decomposition can in turn be used to exhaustively
specify a measurement, as we will see in the next section.

As an example, consider two states $\vert \psi_a\rangle$ and $\vert
\psi_b\rangle$ that are eigenstates of $\bm{A}$ with eigenvalues
$\lambda_a$ and $\lambda_b$, respectively. In the diagonalization
process we have obtained the coefficients $\alpha_0$, $\alpha_1$,
$\beta_0$ and $\beta_1$ using an expansion in terms of the orthogonal
basis $\vert 0\rangle$ and $\vert 1\rangle$.


\paragraph{Explicit results}

We have then

\[
\vert \psi_a\rangle = \alpha_0\vert 0\rangle+\alpha_1\vert 1\rangle,
\]
and
\[
\vert \psi_b\rangle = \beta_0\vert 0\rangle+\beta_1\vert 1\rangle,
\]
with corresponding projection operators

\[
\bm{P}_a=\vert \psi_a\rangle \langle \psi_a\vert = \begin{bmatrix} \vert \alpha_0\vert^2 &\alpha_0\alpha_1^* \\
                                                                   \alpha_1\alpha_0^* & \vert \alpha_1\vert^* \end{bmatrix},
\]    
and
\[
\bm{P}_b=\vert \psi_b\rangle \langle \psi_b\vert = \begin{bmatrix} \vert \beta_0\vert^2 &\beta_0\beta_1^* \\
                                                                   \beta_1\beta_0^* & \vert \beta_1\vert^* \end{bmatrix}.
\]


\paragraph{The spectral decomposition}
The results from the previous slide gives us
the following spectral decomposition of $\bm{A}$
\[
\bm{A}=\lambda_a \vert \psi_a\rangle \langle \psi_a\vert+\lambda_b \vert \psi_b\rangle \langle \psi_b\vert,
\]
which written out in all its details reads
\[
\bm{A}=\lambda_a\begin{bmatrix} \vert \alpha_0\vert^2 &\alpha_0\alpha_1^* \\
                                                                   \alpha_1\alpha_0^* & \vert \alpha_1\vert^* \end{bmatrix} +\lambda_b\begin{bmatrix} \vert \beta_0\vert^2 &\beta_0\beta_1^* \\
                                                                   \beta_1\beta_0^* & \vert \beta_1\vert^* \end{bmatrix}.
\]


\paragraph{First exercise set}

The last two exercises are meant to build the basis for
the two projects we will work on during the semester.  The first
project deals with implementing the so-called
\textbf{Variational Quantum Eigensolver} algorithm for finding the eigenvalues and eigenvectors of selected Hamiltonians.


\paragraph{Ex1: Bell states}

Show that the so-called Bell states listed here (and to be encountered many times in this course) form an orthogonal basis
\[
\vert \Phi^+\rangle = \frac{1}{\sqrt{2}}\left[\vert 00\rangle +\vert 11\rangle\right]=\frac{1}{\sqrt{2}}\begin{bmatrix} 1 \\ 0 \\ 0 \\ 1\end{bmatrix},
\]

\[
\vert \Phi^-\rangle = \frac{1}{\sqrt{2}}\left[\vert 00\rangle -\vert 11\rangle\right]=\frac{1}{\sqrt{2}}\begin{bmatrix} 1 \\ 0 \\ 0 \\ -1\end{bmatrix},
\]


\paragraph{And the next two}
\[
\vert \Psi^+\rangle = \frac{1}{\sqrt{2}}\left[\vert 10\rangle +\vert 01\rangle\right]=\frac{1}{\sqrt{2}}\begin{bmatrix} 0 \\ 1 \\ 1 \\ 0\end{bmatrix},
\]

and

\[
\vert \Psi^-\rangle = \frac{1}{\sqrt{2}}\left[\vert 10\rangle -\vert 01\rangle\right]=\frac{1}{\sqrt{2}}\begin{bmatrix} 0 \\ 1 \\ -1 \\ 0\end{bmatrix}.
\]


\paragraph{Ex2: Entangled state}

Show that the state $\alpha \vert 00\rangle+\beta\vert 11\rangle$ cannot be written as the product of the tensor product of two states and is thus entangle. The constants  $\alpha$ and $\beta$ are both nonzero.


\paragraph{Ex3: Commutator identities}
Prove the following commutator relations for different operators (marked with a hat)
\begin{enumerate}
\item $[\hat{A}+\hat{B},\hat{C}]= [\hat{A},\hat{C}]+[\hat{B},\hat{C}]$;

\item $[\hat{A},\hat{B}\hat{C}]= [\hat{A},\hat{B}]\hat{C}+\hat{B}[\hat{A},\hat{C}]$; and 

\item $[\hat{A},[\hat{B}\hat{C}]]= [\hat{B},[\hat{C},\hat{A}]]+[\hat{C},[\hat{A},\hat{B}]]=0$ (the so-called Jacobi identity).
\end{enumerate}



\paragraph{Ex4: Pauli matrices}
\begin{enumerate}
\item Set up the commutation rules for Pauli matrices, that is find $[\sigma_i,\sigma_j]$ where $i,j=x,y,z$.

\item We define $\bm{X}=\sigma_x$, $\bm{Y}=\sigma_y$ and $\bm{Z}=\sigma_z$. Show that $\bm{XX}=\bm{YY}=\bm{ZZ}=\bm{I}$.

\item Which one of the Pauli matrices has the qubit basis $\vert 0\rangle$ and $\vert 1\rangle$ as eigenbasis? What are the eigenvalues?
\end{enumerate}



\paragraph{Ex5: Shared eigenvectors}

Prove that if two operators $\hat{A}$ and $\hat{B}$ commute they will share a basis of eigenstates


\paragraph{Ex6: One-qubit basis and  Pauli matrices}

Write a function which sets up a one-qubit basis and apply the various Pauli matrices to these basis states.


\paragraph{Ex7: Hadamard and Phase gates}

Apply the Hadamard and Phase gates to the same one-qubit basis states and study their actions on these states.

