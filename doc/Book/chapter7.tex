\chapter{Machine Learning and Bayesian Inference in Quantum Sensing}

\section{Introduction}

Quantum sensing leverages quantum systems (qubits, atoms, defects,
etc.) to measure physical parameters with high precision. Examples
include using quantum probes to measure magnetic fields,
time/frequency standards, or various forces at the nanoscale. The
central goal is to estimate one or more parameters $\theta$ governing
a quantum system’s state or dynamics as accurately as
possible. Traditionally, quantum parameter estimation has been
approached with frequentist methods (e.g. using the quantum Fisher
information and Cramér–Rao bound), but Bayesian inference provides an
alternative framework that is especially powerful when incorporating
prior knowledge or performing adaptive measurements. Frequentist and
Bayesian approaches often agree in the limit of large data, but they
differ conceptually: frequentist precision bounds like the Cramér–Rao
bound generally do not directly apply to Bayesian strategies, and vice
versa . In fact, mixing the two paradigms without care can be
conceptually inconsistent . In this chpater, we introduce
Bayesian estimation theory and its application to quantum sensing,
covering both single-parameter and multi-parameter cases. We discuss
how measurement outcomes (from projective measurements or POVMs) yield
likelihood functions, how Bayes’ theorem updates our knowledge, and
how to quantify the resulting uncertainty (e.g. via credible
intervals, posterior variance, and information gain). We also compare
the Bayesian approach to frequentist quantum estimation (quantum
Fisher information and quantum Cramér–Rao bound), and provide
illustrative examples including nitrogen-vacancy (NV) center
magnetometry, trapped-ion sensors, and superconducting qubit
experiments. Code snippets and figures are included to demonstrate
posterior updates in practice.



\section{Bayesian Estimation Theory Basics}

Bayesian estimation is a statistical inference framework where
probabilities represent degrees of belief (or knowledge) about unknown
parameters. Before any data is collected, one assigns a prior
distribution $p(\theta)$ to represent initial knowledge or uncertainty
about the parameter $\theta$. Upon obtaining data (measurement
outcomes) $D$, Bayes’ theorem is used to update this belief. Bayes’
theorem can be stated as:
\begin{equation}
p(\theta \mid D) = \frac{p(D \mid \theta), p(\theta)}{p(D)},
\label{eq:Bayes}
\end{equation}
where $p(D \mid \theta)$ is the likelihood of observing data $D$ given
$\theta$, and $p(\theta|D)$ is the posterior distribution, i.e. the
updated probability distribution for $\theta$ after taking the data
into account . The denominator $p(D) = \int
p(D|\theta)p(\theta),d\theta$ is the evidence (or marginal
likelihood), ensuring the posterior is normalized. In words, the
posterior is proportional to the likelihood times the prior. This
formula encapsulates how prior beliefs are quantitatively updated by
new evidence.



For a simple example, suppose $\theta$ is an unknown phase and our
prior for $\theta$ is uniform on $[0,2\pi)$. After a measurement
  yielding some outcome data $D$, the likelihood $p(D|\theta)$ will
  favor certain values of $\theta$ over others. Equation
  \eqref{eq:Bayes} tells us how to combine this likelihood with the
  prior to obtain the posterior $p(\theta|D)$, our new state of
  knowledge about $\theta$. The maximum a posteriori (MAP) estimate is
  the value of $\theta$ that maximizes $p(\theta|D)$, while the
  Bayesian mean estimate (the conditional mean,
  $\hat{\theta}{\text{Bayes}} = E[\theta | D]$) minimizes the mean
  squared error (MMSE) among all estimators . Indeed, it is a basic
  result that the posterior mean $E[\theta|D]$ is the Bayes-optimal
  estimator for a scalar parameter under squared-error loss . The
  Bayesian framework naturally encodes a notion of uncertainty in the
  form of the posterior distribution, rather than yielding only a
  point estimate. One can summarize this uncertainty by reporting a
  credible interval (the Bayesian analogue of a confidence interval):
  for example, the central $95\%$ credible interval is an interval
  $[\theta{\text{low}}, \theta_{\text{high}}]$ such that
  $\int_{\theta_{\text{low}}}^{\theta_{\text{high}}}
  p(\theta|D),d\theta = 0.95$. This interval reflects the range within
  which the true parameter lies with $95\%$ posterior probability. An
  important difference from frequentist confidence intervals is that
  credible intervals directly quantify belief about $\theta$; they do
  not rely on hypothetical repeated sampling. Bayesian methods thus
  allow assigning probability to fixed parameter values and provide
  meaningful intervals even after a single experiment . There is no
  concept of an “unbiased estimator” in the Bayesian approach – bias
  is not defined since $\theta$ is treated as a random variable rather
  than an unknown constant . Instead, the focus is on the full
  posterior and its characteristics (mean, variance, etc.), given the
  prior and data.



\section{Quantum Measurements and Likelihood Functions}

In quantum sensing and metrology, the data $D$ comes from measurements
on a quantum system that depends on the parameter(s) of interest. The
link between the parameter $\theta$ and measurement outcomes is
encapsulated by quantum measurement theory. A general quantum
measurement is described by a positive-operator valued measure (POVM),
which is a set ${\Pi_l}$ of positive semi-definite operators on the
state space that sum to the identity. If the quantum system is
prepared in a state $\rho(\theta)$ that depends on the parameter, then
the probability of obtaining outcome $l$ is given by Born’s rule:
\begin{equation}
P(l \mid \theta) = \mathrm{Tr}!\big[\rho(\theta), \Pi_l\big],
\label{eq:Born}
\end{equation}
which serves as the likelihood function $p(D|\theta)$ for Bayesian
inference (with $D$ corresponding to the outcome $l$ in this
measurement) . In the special case of a projective measurement ${\Pi_l
  = |l\rangle\langle l|}$ on an eigenbasis, this reduces to
$P(l|\theta) = \langle l|\rho(\theta)|l\rangle$. The form
\eqref{eq:Born} makes it clear that the choice of measurement
${\Pi_l}$ influences the information we gain about $\theta$. For a
given quantum state family $\rho(\theta)$, some measurements will be
more informative about $\theta$ than others. Bayesian estimation
allows us to incorporate the measurement process directly: we update
the probability distribution for $\theta$ according to the observed
outcomes. If multiple independent quantum measurements are performed
(for example, preparing the state anew each time and measuring), and
one obtains a dataset $D = {l_1, l_2, \dots, l_N}$ of outcomes, the
overall likelihood assuming identical probes is
\[
p(D \mid \theta) = \prod_{i=1}^N P(l_i \mid \theta),
\]
so the posterior after $N$ measurements is proportional to $p(\theta)
\prod_{i=1}^N P(l_i|\theta)$. Sequential Bayesian updating (updating
the posterior one measurement at a time) is particularly natural in
this context, as we will illustrate later.



Optimal Measurements: In frequentist quantum estimation theory, one
quantifies the best achievable precision using the quantum Fisher
information (QFI). The QFI $F_Q(\theta)$ is a property of the family
of states $\rho(\theta)$; it gives the maximal classical Fisher
information obtainable about $\theta$ per quantum measurement,
optimized over all possible POVMs on $\rho(\theta)$. The quantum
Cramér–Rao bound (QCRB) then states that any (unbiased) estimator
$\hat{\theta}$ from $N$ measurements satisfies
$\mathrm{Var}(\hat{\theta}) \ge \frac{1}{N F_Q(\theta)}$ . In the
Bayesian context, where we allow a prior, the analogous optimal
measurement problem was solved by Personick (1971), who found the POVM
that minimizes the Bayesian mean-square error (i.e. optimizes the
posterior variance or other cost) . This Bayesian optimal measurement
generally differs from the one that maximizes QFI, especially for
finite samples or informative priors. However, Personick’s solution is
rarely used directly in practice, because implementing the globally
optimal POVM can be complex. Instead, many practical schemes use
adaptive measurements: at each step, the measurement setting can be
chosen based on the current posterior (thus approximately maximizing
the information gain locally). This approach is a form of Bayesian
experimental design – choose the next measurement to maximally reduce
the expected uncertainty in $\theta$. We will see examples of this
adaptive strategy in later sections.



\section{Single-Parameter vs. Multi-Parameter Estimation}

Quantum sensing scenarios may involve estimating a single scalar
parameter or multiple parameters simultaneously. Single-parameter
estimation is the most common case – for instance, determining a
magnetic field $B$ from the precession of a qubit, or estimating a
phase difference $\phi$ in an interferometer. Bayesian inference
handles this by maintaining a univariate posterior $p(\theta|D)$ that
is updated with each measurement. The formalism is straightforward,
and one can often visualize the posterior as a probability
distribution on $\theta \in \mathbb{R}$ (or a bounded interval).



In multi-parameter estimation, $\boldsymbol{\theta} = (\theta_1,
\theta_2, \dots)$ is a vector of parameters. A concrete example is a
3D magnetometer that wants to estimate $(B_x, B_y, B_z)$
simultaneously using a quantum sensor. Another example in quantum
computing calibration is estimating multiple control parameters
(e.g. laser amplitude and detuning for a trapped-ion gate) at once
. The Bayesian approach naturally generalizes: one maintains a joint
posterior $p(\theta_1,\theta_2,\dots | D)$ over all parameters. In
principle, Bayes’ theorem applies in the same way (with priors and
likelihoods now over multi-dimensional space). However, the curse of
dimensionality can make it computationally challenging to represent
and update a high-dimensional posterior, especially if done naively on
a grid. Techniques like particle filters or Markov Chain Monte Carlo
(MCMC) sampling are often employed for Bayesian inference in higher
dimensions. Moreover, quantum measurements of multiple parameters may
face trade-offs: due to the Heisenberg uncertainty principle or
non-commuting observables, one measurement setting may not be optimal
for all parameters at once. In frequentist terms, the quantum Fisher
information matrix for a multi-parameter problem may not be
simultaneously attainable for all parameters (this is related to
incompatibility of optimal measurements for different
parameters). Bayesian methods can still be applied by treating the
multi-parameter estimation as a joint inference problem, and in
adaptive schemes, one can alternate or choose measurement settings to
balance information gain about the different components of
$\boldsymbol{\theta}$. Recent research has begun to explore Bayesian
multi-parameter quantum metrology, developing adaptive schemes that
extend Personick’s single-parameter optimal strategy to multiple
parameters . For example, an adaptive Bayesian scheme was proposed for
localizing multiple point emitters (a multi-parameter super-resolution
imaging problem), demonstrating better performance than standard
measurement strategies . Similarly, experiments with trapped ions have
shown that one can simultaneously calibrate several control parameters
(e.g. all the parameters of a two-qubit entangling gate) using a
Bayesian algorithm that updates a multi-dimensional posterior and
chooses new measurement settings based on it . The ability to
incorporate prior correlations between parameters and update all
parameters together is a key advantage of the Bayesian approach in
multi-parameter quantum sensing.



\section{Examples of Bayesian Quantum Sensing Applications}



\subsection{NV Center Magnetometry (Single Qubit Sensor)}

Nitrogen-vacancy (NV) centers in diamond are point defects that house
an electron spin ($S=1$) which can be optically initialized and read
out. NV centers are excellent quantum sensors for magnetic fields,
with high sensitivity and spatial resolution down to the nanoscale
. In conventional NV magnetometry, one sweeps a microwave frequency
across a resonance and observes changes in fluorescence to determine
the magnetic field (which shifts the spin transition frequency via the
Zeeman effect). This can be time-consuming. A Bayesian approach can
dramatically speed up this process by focusing measurements where they
are most informative. In a 2020 experiment, Dushenko et
al. demonstrated a sequential Bayesian experimental design for NV
magnetometry . The idea is to treat the unknown magnetic field (or
equivalently the unknown resonance frequency) as a parameter with a
prior, and then iteratively update the posterior after each
measurement. Crucially, after each update, the experimental controller
chooses the next microwave frequency (the next measurement setting) in
real-time to maximize the expected information gain (“utility”) based
on the current posterior . By doing so, the experiment “zooms in” on
the resonance much faster than a blind sweep. The reported result was
more than an order of magnitude reduction in measurement time to reach
a given precision, compared to traditional methods . This Bayesian
adaptive magnetometry is effectively an instance of active learning:
the experiment learns the parameter as it goes, and intelligently
focuses on the most informative measurements. It highlights the power
of Bayesian methods in quantum sensing – especially when data
acquisition is costly or time-limited, an adaptive strategy can
extract the most information with the fewest measurements.



\subsection{Trapped Ion Sensors and Calibration}

Trapped ions are among the most precise quantum systems, used in
optical atomic clocks (for time/frequency measurement), as inertial
sensors, and in quantum computing platforms that require careful
calibration of laser and magnetic field parameters. Bayesian inference
has found use in these contexts in two main ways: (1) Sensing external
parameters (like frequencies or forces), and (2) Calibrating internal
operation parameters of a quantum device. An example of the first kind
is an atomic clock: one can use a Bayesian filter to lock a laser
frequency to an ion’s transition by continuously updating the belief
about the frequency offset, achieving optimal tracking of drifts. An
example of the second kind is given by Gerster et al. (2022), who
developed an experimental Bayesian calibration protocol for two-qubit
entangling gates in a trapped-ion quantum processor . In their
approach, multiple control parameters (like laser detuning, amplitude,
etc.) are estimated simultaneously using a Bayesian algorithm. After
each experiment (a sequence of operations whose success/fidelity
depends on these parameters), the algorithm updates a
multi-dimensional posterior over the parameter space. Crucially, it
uses an adaptive strategy to decide which experiment or measurement to
do next, based on the current posterior, and it stops automatically
once the desired precision (gate fidelity) is reached . This automated
Bayesian tune-up achieved a reliable calibration in under one minute –
a significant speedup over traditional manual calibration . Even
though this example comes from quantum computing, it is essentially a
multi-parameter quantum sensing scenario (sensing/calibrating the
“effective” Hamiltonian parameters of the system). Another trapped-ion
example is Bayesian phase estimation for interferometry: for instance,
de Neeve et al. (2023) implemented a Bayesian phase lock on a single
trapped ion to track a time-varying phase with high accuracy, using an
adaptive phase measurement schedule. These applications show that
Bayesian methods are well-suited to systems where measurements are
sequential and can be adjusted on the fly, and where prior information
(or intermediate data) can be leveraged to optimize the experiment.



\subsection{Superconducting Qubits and Quantum Devices}

Superconducting qubits are a leading platform for quantum computing
and are also used as sensitive detectors of, e.g., microwave photons
or magnetic flux. In the context of quantum sensing and device
calibration, Bayesian inference has been used to tackle noise and
drift in superconducting qubit systems. For example, Sung et
al. (2023) demonstrated a real-time Bayesian estimation approach for
calibrating the frequency of a transmon qubit that experiences slow
drift . They implemented an adaptive algorithm (nicknamed “frequency
binary search”) on a classical FPGA controller: the controller
performs a Ramsey experiment on the qubit with a carefully chosen
evolution time, observes the outcome, and then updates a probability
distribution (over possible qubit frequency values) on the fly . Each
subsequent Ramsey experiment is chosen to maximally reduce the
uncertainty (the algorithm essentially splits the probability
distribution at its median, akin to a binary search in frequency
space) . Using this Bayesian feedback loop, the qubit frequency drift
was tracked in real-time and corrected via feedback to the qubit’s
control frequency. This resulted in improved qubit coherence and gate
fidelity, as the system could be kept closer to its optimal operating
point despite fluctuations . The Bayesian approach in this case has
advantages over a traditional frequentist calibration (which might
repeat fixed sequences and average results): by always using the
current “best guess” of the frequency, the calibration achieves high
precision with fewer measurements and can react promptly to
changes. More generally, superconducting qubit experiments have
embraced Bayesian techniques for tasks like qubit state tomography
with imperfect measurements (using Bayesian updates to estimate the
state or error rates) and quantum error mitigation (inferring error
parameters via Bayes). As quantum hardware scales up, fast Bayesian
calibration routines are becoming important to maintain performance
with reasonable calibration time budgets .



\section{Bayesian Updating and Credible Intervals}

A core feature of Bayesian inference is the cycle of posterior
updating. After each measurement, the prior distribution is updated to
the posterior via Bayes’ rule; this posterior can then serve as the
prior for the next measurement. In practice, one often starts with a
broad prior (reflecting high initial uncertainty), and as more data
accumulate, the posterior typically becomes sharper (lower variance),
concentrating around the true parameter value. This iterative
refinement embodies the learning process. The following pseudocode
illustrates the basic Bayesian update loop for a generic
parameter-estimation experiment:



\begin{verbatim}





Initialize prior (discrete grid or analytic distribution)





prior = initial_prior_distribution

for each measurement i = 1 to N:

choose measurement setting (e.g. pulse duration, basis angle) based on prior

obtain outcome k_i from quantum measurement

compute likelihood L(theta) = P(k_i | theta)  for all theta

posterior = L(theta) * prior   (multiply prior by likelihood pointwise)

normalize posterior

prior = posterior  (use current posterior as prior for next iteration)

\end{verbatim}



After each update, one can compute summary statistics of the
posterior. For example, the posterior mean gives the current best
estimate of $\theta$, and the posterior variance quantifies the
remaining uncertainty. As mentioned, one can also extract a credible
interval: e.g., find an interval $[\theta_{\text{low}},
  \theta_{\text{high}}]$ that contains, say, $95\%$ of the posterior
probability. This credible interval shrinks as more data are acquired
(for well-behaved likelihoods), reflecting increased confidence in
$\theta$’s value. Figure \ref{fig:posterior_evolution} below shows an
example of this updating process for a simulated phase estimation
experiment.



Figure \label{fig:posterior_evolution}: Evolution of the posterior
distribution over a single unknown phase $\phi$ (in degrees) as
measurements are performed. The initial prior is uniform
(top-left). After 2 measurements (top-right), the distribution begins
to concentrate around two symmetric high-probability regions (arising
from ambiguity in the measurement results). After 4 measurements
(bottom-left), one of the two candidate regions has been largely ruled
out, and the posterior concentrates near the true phase. After 8
measurements (bottom-right), the posterior is sharply peaked,
indicating high confidence in the phase estimate. In this simulation,
each measurement was a single-qubit Ramsey experiment with outcomes
used to update the Bayesian estimate of $\phi$. All eight outcomes in
this particular run happened to be consistent with the true phase
$\approx 45^\circ$, resulting in a posterior mean essentially equal to
the true value. The narrowing of the distribution illustrates how
additional data improve precision.



In addition to point estimates and credible intervals, one can
quantify the information gained from data. One measure of information
gain is the reduction in the entropy of the distribution: for example,
the Shannon entropy of the prior $H[p(\theta)]$ minus that of the
posterior $H[p(\theta|D)]$. Bayesian adaptive methods often choose the
next measurement to maximize the expected reduction in entropy (or
equivalently, maximize the expected Kullback–Leibler divergence
between posterior and prior), which is the same as maximizing the
expected information gain. Another closely related metric is the
posterior variance: a smaller posterior variance means more
information (narrower distribution). In fact, some adaptive algorithms
are designed to greedily minimize the expected posterior variance at
each step . In the superconducting qubit frequency-tracking example,
the “binary search” strategy was essentially minimizing the worst-case
posterior variance by bisecting the probability mass of the current
distribution .



Bayesian methods also allow real-time assessment of when enough data
have been collected. Because the posterior at each step is an entire
distribution, one can set a stopping criterion based on the
uncertainty: for instance, stop measuring when the posterior standard
deviation falls below a target threshold, or when the $95\%$ credible
interval width is below some requirement. This was implemented in the
trapped-ion gate calibration example, where the algorithm stopped
automatically once the desired gate infidelity (which is a function of
the parameter uncertainty) was reached . Such criteria are part of
Bayesian experimental design – balancing the cost of more data against
the benefit of improved precision.



\section{Posterior Variance and Information Gain}

The posterior variance $\mathrm{Var}[\theta | D]$ is a key figure of
merit for Bayesian estimators. It quantifies the expected squared
deviation of $\theta$ from its mean given the data, i.e. how uncertain
we remain about $\theta$ after incorporating the evidence. Posterior
variance is directly related to the MMSE (mean squared error) of the
Bayesian estimator (since the Bayes estimator is typically the
posterior mean under squared-error loss). We often seek to minimize
this variance with optimal experiment design. Each new measurement can
be viewed as reducing the variance on average – but by how much? This
is where information gain comes in. The expected information gain from
a measurement (before doing it) can be defined as the difference
between the prior and expected posterior entropy, or equivalently as
the expected reduction in variance or other uncertainty
quantifier. Many adaptive Bayesian protocols choose the next
measurement setting $x$ by maximizing the expected information gain:
\[
x^* = \arg\max_x \; \Big[ \mathbb{E}{k \sim p(k|\text{current posterior},\,x)} \big[ I{\text{prior}} - I_{\text{posterior after }k} \big] \Big],
\]
where $I$ is an information measure (like entropy or variance) and the
expectation is over possible outcomes $k$ of the next
measurement. This abstractly represents choosing the setting that, on
average, will yield the most informative result. In practice, one
might discretize possible measurement settings and try each in
simulation (using the current posterior as a surrogate of truth) to
evaluate the expected posterior variance. The NV center experiment
described earlier effectively used such a utility function (though the
details were described in terms of a “utility” rather than explicitly
in terms of bits of entropy) . The superconducting qubit calibration
used a strategy that is locally optimal in that it halves the
distribution’s support in frequency space at each step, minimizing the
worst-case posterior spread .



It is worth noting that there is a Bayesian analogue of the Cramér–Rao
bound called the Bayesian Cramér–Rao bound (or Van Trees
inequality). This bound provides a lower limit on the posterior
variance (or more generally, on the mean squared error) given a prior
and likelihood model. It essentially combines prior uncertainty and
likelihood Fisher information to bound the achievable
precision. However, the Van Trees bound is rarely saturated in
practice; it serves more as a benchmark. In some cases, one can show
that as the prior becomes very broad (un-informative) and the number
of measurements $N$ becomes large, the Bayesian approach approaches
the frequentist limit. Specifically, the posterior tends towards a
normal (Gaussian) distribution centered at the true parameter value,
with variance approximately $1/(N F_{\text{cl}})$ where
$F_{\text{cl}}$ is the classical Fisher information of the chosen
measurement strategy . In this asymptotic regime, Bayesian credible
intervals at (say) $95\%$ credibility will coincide with the $95\%$
confidence intervals from frequentist analysis, and the Bayesian
estimator’s variance will approach the Cramér–Rao bound . But outside
of the asymptotic limit, and especially for adaptive schemes or
informative priors, the Bayesian variance can be lower than the naive
frequentist CRB computed for an unbiased estimator . This is not a
violation of any principle; it stems from the fact that the Bayesian
scenario (with a strong prior or a specific adaptive measurement
sequence) is a different problem than the one the frequentist CRB is
addressing . In summary, posterior variance gives a concrete handle on
uncertainty in the Bayesian picture, and maximizing information gain
is the guiding principle to reduce it as rapidly as possible in
quantum sensing tasks.



\section{Comparison with Frequentist Methods}

It is insightful to contrast the Bayesian approach to quantum
parameter estimation with the more traditional frequentist approach
based on the quantum Fisher information (QFI) and quantum Cramér–Rao
bound (QCRB). In the frequentist paradigm, the parameter $\theta$ is
considered a fixed but unknown constant. One typically imagines
repeating an experiment many times and constructing an estimator
$\hat{\theta}(D)$ as a function of the data. The Quantum Fisher
Information $F_Q(\theta)$ is a property of the quantum state family
$\rho(\theta)$ that sets an upper limit on the achievable classical
Fisher information for any measurement . The QCRB states
$\mathrm{Var}(\hat{\theta}) \ge 1/(N F_Q(\theta))$ for any unbiased
estimator using $N$ independent, identical probes . The power of the
frequentist approach is that it gives a crisp benchmark: for example,
it tells us that no matter what (unbiased) data analysis you do, you
cannot beat that variance bound with $N$ measurements. It also
prescribes, in principle, what measurement to perform – one that
saturates the QFI (often the eigenbasis of the symmetric logarithmic
derivative operator). However, this approach has limitations. Notably,
the CRB is only an equality in the limit of large $N$ under certain
regularity conditions (e.g., the likelihood must be well-behaved and
the estimator unbiased). For finite samples, the CRB might not be
tight, and for biased estimators or for Bayesian strategies (which
incorporate priors), the CRB need not apply at all .



Bayesian estimation, on the other hand, does not require the concept
of an unbiased estimator or repeated trials in the same sense. By
incorporating prior information, Bayesian methods can achieve better
precision with fewer data points in scenarios where a good prior is
available. Even with a flat (ignorance) prior, a Bayesian adaptive
strategy can outperform a fixed strategy that the frequentist analysis
assumes. For example, in the quantum phase estimation context,
frequentist analysis might say that the best you can do (with $N$
photons or $N$ runs) is the standard quantum limit or Heisenberg limit
depending on the state and measurement, but a Bayesian adaptive phase
estimation protocol can achieve that scaling and often saturate the
bound with fewer resources by dynamically adjusting measurements
. Pezzè et al. (2018) explicitly compare frequentist and Bayesian
phase estimation, noting that frequentist CRB-style bounds do not
directly apply to Bayesian strategies and vice versa . They also show
examples where a Bayesian credible interval can be narrower than the
nominal frequentist confidence interval in certain regimes – which is
not a contradiction, but a reflection of the additional information
(or different criteria) being used.



Another point of comparison is how each framework treats multiple
parameters. In frequentist quantum estimation, the QFI generalizes to
a QFI matrix for a vector parameter $\boldsymbol{\theta}$, and one
obtains a matrix-bound on the covariance of any unbiased estimator:
$\mathrm{Cov}(\hat{\boldsymbol{\theta}}) \ge (N
F_Q(\boldsymbol{\theta}))^{-1}$ (as a matrix inequality). However,
when the optimal measurements for different components do not commute,
one cannot simultaneously saturate all components of this bound. The
Bayesian approach naturally sidesteps this by focusing on the overall
error and allowing trade-offs via the prior and loss function. One
can, for instance, assign a cost function that weights different
parameters and directly minimize the Bayes risk for that cost; the
“best” measurement might then be a compromise that no single-parameter
CRB would capture.



In summary, frequentist methods provide powerful analytic bounds and
design principles (especially useful for determining the ultimate
limits of precision in a given physical scenario), while Bayesian
methods offer a flexible, adaptive way to actually achieve high
precision in practice, especially when prior knowledge or on-the-fly
adjustment is possible. Both approaches often converge in the limit of
many measurements – e.g., a Bayesian method with a flat prior will
produce a posterior that (by the Bernstein–von Mises theorem) is
approximately Gaussian around the true value with width $1/\sqrt{N
  F_Q}$, consistent with the CRB . But for finite $N$ and with prior
information, the Bayesian approach can be more powerful and
informative. As one reference succinctly put it, “frequentist
precision bounds, such as the Cramér–Rao bound, do not apply to
Bayesian strategies, and vice-versa” – they are different ways of
framing the problem. A prudent quantum engineer should understand
both: use frequentist QFI analysis to know what is theoretically
possible, and use Bayesian techniques to design practical protocols
that approach those ideals under realistic conditions.



\section{Illustrative Example: Bayesian Phase Estimation Simulation}

To solidify the concepts, let us consider a simple quantum phase
estimation scenario and walk through a Bayesian inference
simulation. Suppose we have a single qubit serving as an
interferometer: it accumulates a phase $\phi$ (the parameter to be
estimated) between two $\pi/2$ pulses, and we can measure the qubit in
a basis that has some adjustable phase reference. Specifically, assume
that if we measure the qubit in a basis rotated by an angle $\alpha$
(relative to the $Z$-basis), the probability of obtaining outcome
“$0$” (projecting onto the state $(|0\rangle +
e^{i\alpha}|1\rangle)/\sqrt{2}$) is $P(0|\phi,\alpha) = \frac{1}{2}[1
  + \cos(\phi - \alpha)]$. This is akin to a Ramsey interferometry
fringe: by changing the analysis phase $\alpha$, we can probe
different quadratures of the phase. Now, the task is to estimate an
unknown $\phi$.



For this simulation, let the true phase be $\phi_{\text{true}} =
50^\circ$ (approximately $0.87$ radians). We start with a uniform
prior on $\phi \in [0,360^\circ)$. We will perform a sequence of
  measurements. A reasonable strategy here is to alternate between
  $\alpha=0^\circ$ and $\alpha=90^\circ$ measurements, which
  correspond to measuring $\cos\phi$ and $\sin\phi$ fringes,
  respectively. (In practice, more sophisticated adaptive choices
  exist, but we use a simple fixed sequence for illustration.) We
  generate synthetic measurement outcomes from the true $\phi$ and
  update the posterior after each one. The plots in
  Fig. \ref{fig:posterior_evolution} (shown earlier) were generated
  from one such simulation. All eight measurements in that run were
  favorable (the qubit outcomes were all $0$, which statistically is
  the most likely outcome for $\phi_{\text{true}}=50^\circ$ and our
  chosen measurements, as $\cos 50^\circ$ and $\sin 50^\circ$ are
  positive). Consequently, the posterior homed in quickly on the true
  value. In other runs, one might get some outcomes as $1$, which
  would shift the distribution differently – but on average the
  distribution will still converge around the correct phase.



This example can be implemented with a few lines of code in Python,
using a discrete grid to represent $p(\phi)$. Below is a simplified
version of such a code (using degrees for simplicity and an 8-bin
discretization just for demonstration):


\section{Conclusion}
